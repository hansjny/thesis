\chapter{Introduction}
The presence of Wireless LANs is ever expanding, and by 2020 it is esitmated that we have 20 billion connected devices \cite{Gartner}.
According to the Wi-Fi Alliance, households alone will be home to over 10 billion devices that are able to connect to
the home router \cite{wifialliance}. 
Even while introducing more and more devices to a home in the shape of smart TVs, lights, and audio systems, at the same time
a user expects the Wireless LAN connection to deliver optimal speeds to meet our time's demand for universal wireless access. 
Not only is the number of devices growing, but our traffic patterns has changed. In private homes
continous streams of high bitrate video and audio is highly common, and it is not unusual that these flows happen
simultaneously in a family home. Also as network-attached storage has become available to consumers,
and the popularity of cloud storage is also increasing, the new traffic that is transmitted over Wireless LANs
often also includes data transmissions and backups that may happen in a background process. 
While the physical layer (PHY) traditionally has been upgraded to meet the new demands in bitrate (e.g. fiber over wire, 
and 802.11ac) Wireless LAN connections struggle under the heavy impact traffic intereference which can not be solved by increasing
the physical datarate capacity. 

\section{Motivation}
Wireless LAN is deployed in almost all corporate buildings and residencies in the western world, and increasingly also in the rest of the world.
The use of these networks used to be limited to laptops that generated small amounts of data, but now the range of devices includes
smart-phones, network storage devices, even in some cases servers. The deployment of Wireless LAN and the infrastructure has not 
changed much over the years to match the new demands and increased traffic, and in most places coverage is still the main concern, while
Quality of Service (QoS) comes second. There has been done much research on how centralized controllers can benefit the deployment
of enterprise networks (\cite{Murty}, \cite{Murty2} and \cite{Suresh}),
but in this thesis we will address the issues related to deployment to Wireless LANs in residiential areas. Customers who are subscribed
to high data rate service level agreements often find they can only receive a comparable data rate over wired LAN. When the
Wireless LAN networks in their home is constantly underperforming, it is not unusual for an unknowing customer to upgrade the
data rate of their agreement - to no effect. This leads to customers being largely frustrated with their Internet service providers,
even though they are not at fault. In many scenarios a customer service representative or a tech-savvy customer may manually
switch the operating channel for a wireless access point. If this has no effect a customer might be encouraged to get a router
which can transmit a more powerful signal. While this might prove to be a quick fix for the customer who was resourceful enough to deal with the problem,
it in turn may trigger a chain reaction of even stronger interference levels for the rest of the inhabitants in the surrounding area.
In this thesis we are going to look at a possible way for Wireless LAN access points in residential homes to organize themselves
in groups and communicate with each other to collaborate on channel assignment.

\section{Problem definition}
There are 3 non-overlapping channels on the 2.4GHz spectrum that 802.11 Wireless LANs uses. One of the usual ways of selecting a channel
is done by letting an access point sense which channel has the lowest interference levels. When channels are selected this
selfishly it is very unlikely that the distribution of channels in a confined area (e.g. an apartment block) becomes optimal.  
Ideally all access points would be configured so the channel allocation is optimal for an entire area.
This can be realized with a centralized controller, but in residential WiFi networks there are by default no centralized controller.
We will explore how wireless access points can organize themselves in confined groups that collaborate on selecting an optimal channel
distribution for the entire group to maximize the efficiency of wireless networks.

\section{Method}
To be able to see if we can form groups in a way that creates clusters of nearby nodes we need to get some data to perform
calculation on. We will both create syntethic data and use real world location data of access points to evaluate the performance of the group creations. 
Then we will take a look at possible solutions to let access points communicate with each other without any bootstrapping or 
previous association. Then we will consider the problems and challenges that has to be overcome in the process of creating and deploying 
an architecture as suggested in the thesis. 

    %We will take a look on earlier algorithms in the research of finding a better way to allocate channels in 802.11. 
    %Then we can do a number of assumptions to be able to propose an algorithm for group creation amongst unorganized access points,
    %and then evaluate the algorithm by computing groups and clusters of access points.  




%\section{Channel allocation} 
%To deal with the problem of channel allocation we will think of an AP as a vertex in a graph. When an AP scans its radio
%it can hear the strength of all nearby wireless networks measured in dBm (decibel milliwatts). This decibel value will be
%the value of the edge between one AP to another. With a graph expressing the wireless network topology, the problem
%of optimally distributing channels between APs boils down to a graph coloring problem. The number of colors in the color problem,
   %represents the number of non-overlapping channels in 802.11. Exactly how an algorithm can be designed to optimally distribute channels within the
   %interfering topology is out of the scope of this thesis. However we can define some invariants that has to be true
   %for such an algorithm to work:
   %\begin{enumerate} 
   %\item All APs has to run the same algorithm
   %\item All APs must run the algorithm on the same connected group
   %\item Because of the complexity of the problem the algorthm must solve, the number of APs in the connected group can not be too big
   %\end{enumerate}
%
   %Point 1 is trivial to solve or mitigate, as only APs running the algorithm will actively participate in the channel selection. A simple way to make sure that the
   %same algorithm is used, is by having a software version that is consistently checked with the other APs in the connected group.
%
   %Point 2 and point 3 is will be the main focus of the rest of the master thesis, as these are not so easily solved.
%
   %We can define a wireless topology graph as a set of wireless APs that are grouped together and share information about their neighbours and interference levels.
   %This set is what will now on be referred to as a \textit{connected group.} All members of the connected group will be considered when running the channel assignment algorithm.
   %For the connected group to have an actual impact on the quality of a network connection, it has to consist of nodes that normally disturbs each other substantially.
%
   %An ideal example of a connected group is an apartment builiding. The channel allocation protocol lets APs share information about who-disturbs-who the most in the building.
   %Then each AP can run the channel allocation algorithm. Because they run it on the same graph, every AP will find the same optimal channel distribution throughout the building,
   %and then switch to the correct channel. 
%
   %Even though an apartment building is most likely an optimal delimination of a connected group, in reality creating such a group is a bigger challenge. As the whole channel allocation
   %protocol is based on decentralized peer-to-peer technology, and no centralized server with access to demographical and geographical divisions exists, the protocol will
   %have to discover suitable connected groups on its own. Moreover, when the group is created the protocol will have to replicate data so that
   %all participants of the group has all the data required to perform channel allocation. It will also need a way to make sure that the image of the current group
   %is consistent within all APs in the connected group. 


