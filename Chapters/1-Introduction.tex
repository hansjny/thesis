\chapter{Introduction}
The amount of Internet connected devices has for the past few years been rapidly increasing, and is still increasing this day. The latest forecasts
esitmates that there will be about 27 billion devices connected to the Internet by 2021 \cite{CiscoVNI2017}, while by the end of the 2017 the number
connected devices will be close to 8.4 billion \cite{Gartner}. In 2020 households alone will be responsible for over 10 billion devices
able to wirelessly connect to the home router \cite{wifialliance}, and wireless traffic will account for 63 percent of all IP-traffic in the world \cite{CiscoVNI2017}. 
Traditional devices connected to the Internet like computers, phones, watches, smart TVs, audio systems, and lately also private network storage systems is responsible
for much of the traffic. However, as the era of Internet of Things (IoT) has rapidly descended upon us, increasingly also less obvious utilites are connected to the Internet.
These devices may span everything from lights and HVAC systems to coffee machines, fridges and even toasters. Whereas in the early 2000s it was common to own 
one or two MAC-addresses per person, at the date of writing it is not unusual to possess a two digit number of MAC-addresses.

But it is not only the numbers of devices that has changed. The consumers' expectations and demands are also being altered over time. 
While ubiquitous connectivity is a buzzword often heard in the context of future 5G networks, ubiquitous access is already expected in
modern households. The demand for Wi-Fi coverage extends through the entire home: the Internet radio in the garage, the video stream in the basement
couch and the gaming laptop in the bedroom. All demands coverage and expects mobility at the same time. Many of these devices also have something else in
common which reflects the change that has happened over the last few years: demand for high data rate. In 2016 about 73 percent of all IP traffic orginated from video
streams, and in 2021 this number is expected to increase all the way up to 82 percent \cite{CiscoVNI2017}. 

The new demand for continous streams of high bitrate video poses a challenge that can not simply be solved by providing larger bandwidth. First of all, video streams
have high Quality of Service (QoS) demands. Naturally, buffering of videos and/or reduction of video quality will negatively impact the experience of the service.
Secondly, it is not unusual that different video flows occur simultaneously in the same household. If such traffic is transmitted wirelessly
it results in an inherently busy transmission medium. Because of the international spectrum allocation standards, Wi-Fi is largely restricted to a small number of channels.
To overcome this challenge, the 802.11 protocol specifies a set of rules which allows only one device in the vicinity to transmit on a channel at the same time. 
This presents a largely unresolved challenge of mapping channels to access points in a way that there are only a few, or better yet no, devices on the same channels adjacent
to each other.

While the physical layer traditionally has been upgraded to meet the new demands in bitrate (e.g. fiber to the home, 
and MIMO in 802.11ac) Wireless LAN connections struggle under the heavy impact of radio frequency intereference which can not be solved by increasing
the physical datarate capacity.

There has been done research on how centralized controllers can benefit the deployment of large enterprise networks (\cite{Murty}, \cite{Murty2} and \cite{Suresh}),
but in this thesis we will address the emerging issue of deployment of Wi-Fi in residential areas. More specifically, we will consider possible clustering methods
to enable routers and access points to organize themselves in self managing groups. A synchronous, distributed group across access points in different service sets
would allow for planned, cooperative channel allocation. This would transform the channel allocation problem from finding a local optimal channel for each router,
to a problem of finding an optimal channel plan for the entire group.

\section{Motivation}
Wireless LAN, commonly referred to as Wi-Fi, is deployed in almost all corporate buildings and residencies in the modern world.
The use of these networks used to be limited to laptops or phones that generated small amounts of data, but now the range of devices includes
smart-phones, network storage devices, and IoT appliances. The deployment of Wi-Fi and the infrastructure has not 
changed much over the years to match the new demands and increased traffic. Actually, in most places coverage is still the main concern, while
QoS comes second. 

Customers subscribed to high data rate service level agreements often find they can only receive a comparable data rate over wired LAN.
When the Wi-Fi network in their home is constantly underperforming, it is not unusual a customer to upgrade the data rate of their agreement.
But as it happens, often interfering networks are the perpetrators, and an increase of bandwith will have no effect.
This can lead customers to frustration with their Internet service providers, even though it is the underlying technology and not the service provider which is at fault.

There exists a large amount of different algorithms that deal with channel allocation to prevent and limit interference. Some even consider the QoS observed by the client - not only the access points.
However, many efficient algorithms are deployed in centralized systems and does not focus on decentralized instances, such as residential networks.
A decentralized, distributed solution to optimize channel distribution would be helpful in apartment buildings and other residential zones where the density of wireless networks is high,
as it would not require the wireless networks to be under the administration of the same controller (e.g. an ISP). The access points could communicate and organize channel allocation
on their own. Of all channel allocation algorithms only a few addresses the issue of optimizing the channel distribution,
but the algorithms does not have a way to limit the amount of nodes to consider.  The main motivation behind this thesis is to explore ways to create a distributed group (clustering) scheme. Once a group is established, many problems can be solved with technologies previously only used in centralized systems.

\section{Problem definition}
There are 3 non-overlapping channels on the 2.4GHz spectrum utilized by 802.11 Wi-Fi. Today, one of the more common ways of selecting a channel
is done by letting an access point sense which channel has the lowest interference levels, also called least congested channel search.
When channels are selected in this selfish manner, where the only available information about the surrounding networks are obtained via
local radio observations, it is highly unlikely that the channel distribution in a confined area (e.g. an apartment block) becomes optimal.  

Ideally we want all access points to be configured so the channel distribution in an area leads to as few equal channels as possible
being next to one another. The task of optimizing channel distribution is essentially what is called a graph-coloring problem, where the problem is
finding a distribution where no adjacent nodes should operate on the same channels. This is NP-hard and can be solved with heuristics \cite{Brelaz}, but is not the focus of this thesis. 

Since the problem is NP-hard, it is easier to determine a good channel distribution if fewer access points are considered. In other words, if the groups are too large, 
the channel distribution problem becomes too complicated to solve. This bring us to the punch line of the problem definition: finding a way to confine access points to a high-impact group.
We have already established the motivation behind aiming for a decentralized solution - but to achieve this, access points have to find these groups on their own.
It is easy to intuitively say that, for instance, an apartment building should be one group. It is less obvious to answer how all access points in an apartment building
would identify the boundaries of the building and create a group on their own, entirely independent of an Internet service provide (and ideally also router brand).
It boils down to a distributed clustering problem, where no access point has a global view of the landscape of neighbouring routers. So we must find a way to make the clustering algorithm
able to be run in a way that clusters can be built relying only on information that can be obtained from radio scans of its membering access points.

\section{Method}
To be able to see if we can form groups in a way that creates clusters of nearby nodes we need to get some data to perform
calculation on. We will both create syntethic data and use real world location data of access points to evaluate the performance of the group creations. 
Then we will take a look at possible solutions to let access points communicate with each other without any manual bootstrapping or 
previous association. Then we will consider the problems and challenges that has to be overcome in the process of creating and deploying 
an architecture as suggested in the thesis. 

    %We will take a look on earlier algorithms in the research of finding a better way to allocate channels in 802.11. 
    %Then we can do a number of assumptions to be able to propose an algorithm for group creation amongst unorganized access points,
    %and then evaluate the algorithm by computing groups and clusters of access points.  




%\section{Channel allocation} 
%To deal with the problem of channel allocation we will think of an AP as a vertex in a graph. When an AP scans its radio
%it can hear the strength of all nearby wireless networks measured in dBm (decibel milliwatts). This decibel value will be
%the value of the edge between one AP to another. With a graph expressing the wireless network topology, the problem
%of optimally distributing channels between APs boils down to a graph coloring problem. The number of colors in the color problem,
   %represents the number of non-overlapping channels in 802.11. Exactly how an algorithm can be designed to optimally distribute channels within the
   %interfering topology is out of the scope of this thesis. However we can define some invariants that has to be true
   %for such an algorithm to work:
   %\begin{enumerate} 
   %\item All APs has to run the same algorithm
   %\item All APs must run the algorithm on the same connected group
   %\item Because of the complexity of the problem the algorthm must solve, the number of APs in the connected group can not be too big
   %\end{enumerate}
%
   %Point 1 is trivial to solve or mitigate, as only APs running the algorithm will actively participate in the channel selection. A simple way to make sure that the
   %same algorithm is used, is by having a software version that is consistently checked with the other APs in the connected group.
%
   %Point 2 and point 3 is will be the main focus of the rest of the master thesis, as these are not so easily solved.
%
   %We can define a wireless topology graph as a set of wireless APs that are grouped together and share information about their neighbours and interference levels.
   %This set is what will now on be referred to as a \textit{connected group.} All members of the connected group will be considered when running the channel assignment algorithm.
   %For the connected group to have an actual impact on the quality of a network connection, it has to consist of nodes that normally disturbs each other substantially.
%
   %An ideal example of a connected group is an apartment builiding. The channel allocation protocol lets APs share information about who-disturbs-who the most in the building.
   %Then each AP can run the channel allocation algorithm. Because they run it on the same graph, every AP will find the same optimal channel distribution throughout the building,
   %and then switch to the correct channel. 
%
   %Even though an apartment building is most likely an optimal delimination of a connected group, in reality creating such a group is a bigger challenge. As the whole channel allocation
   %protocol is based on decentralized peer-to-peer technology, and no centralized server with access to demographical and geographical divisions exists, the protocol will
   %have to discover suitable connected groups on its own. Moreover, when the group is created the protocol will have to replicate data so that
   %all participants of the group has all the data required to perform channel allocation. It will also need a way to make sure that the image of the current group
   %is consistent within all APs in the connected group. 


