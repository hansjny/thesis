\chapter{Group communication and state synchronization}
In the previous chapters we have suggested an algorithm for how groups can be formed.
In this chapter we will focus on the algorithms and protocols required to enable communication between access points and ultimately groups.
There has already been done some work on the subject of access point communication over IP, and we can also rely on previous work in the field
of distributed consensus to ensure synchronized group states. We will consider these technologies and suggest a protocol that stitches all required components together. 


\section{Problem overview}
In chapter \ref{chap:clustering} we looked at an algorithm that enables cluster formation in a distributed environment where no node knows the complete layout of the surrounding networks.
A number of assumptions were made before we suggested an algorithm. Two of the assumptions encapsulated the three following problems:
\begin{itemize}
\item Direct contact between access points is possible
\item An underlying group communication protocol is in place
\item The state of a group is synchronized throughout all its members
\end{itemize}
In this chapter we shall attempt to handle these issues to create a more complete scheme for group creation in a distributed environment.

\section{Enabling technologies}
When creating the protocol, there is need for some some already well-researched technology as a foundation. Raft \cite{raftio} is a technology created to simplify distributed consensus, and
ResFi is a technolohy create to enable communication over IP between access point. This section will briefly introduce these. 
\subsection{Distributed consensus with Raft}
\subsection{Access point communication with ResFi}

\section{Protocol design}

\section{Protocol implementation}
