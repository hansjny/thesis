\chapter{Introduction}
The presence of Internet connected devices is still rapidly expanding. The latest forecast 
esitmates that there will be about 27 billion devices connected to the Internet by 2021 \cite{CiscoVNI2017}, while by the end of the 2017 the number
connected devices in use will be 8.4 billion \cite{Gartner}. In 2020 households alone will be responsible for over 10 billion devices that are
able to wirelessly connect to the home router \cite{wifialliance}, and wireless traffic will account for 63 percent of all IP traffic in the world \cite{CiscoVNI2017}. 
Traditionally devices that are commonly connected to the Internet in a household are computers, phones, watches, smart TVs, audio systems, and lately also private network storage systems.
As the era of Internet of Things (IoT) has rapidly descended upon us, increasingly also less obvious utilites are connected to the Internet.
These devices may span everything from lights and HVAC systems to coffee machines, fridges and even toasters. Whereas in the early 2000s it was common to own 
one or two MAC-addresses per person, at the date of writing it is not unusual to possess a two digit number of MAC-addresses.

But it is not only the numbers of devices that has changed. The consumers' expectations and demands are also being altered over time. 
While ubiquitous connectivity is a buzzword often heard in the context of future 5G networks, ubiquitous access is already expected in
modern households. The demand for Wi-Fi coverage extends through the entire home: the Internet radio in the garage, the video stream on the basement
couch and the gaming laptop on the bedroom. All demands coverage and expects mobility at the same time. These devices also have something else in
common that reflects the change that has happened the last years: demand for high data rate. In 2016 about 73 percent of all IP traffic orginated from video
streams, and in 2021 this number is expected to increase all the way up to 82 percent \cite{CiscoVNI2017}. 

This demand for continous streams of high bitrate video poses a challenge that can not only be solved by providing larger bandwidth. First of all, video streams
have high Quality of Service (QoS) demands. Buffering of videos and reduction of video quality negatively affects the subjective experience.
Secondly, it is not unusual that different video flows occur simultaneously in a single residence. If such traffic is transmitted wirelessly
it results in an inherently busy transmission medium. Because of the international spectrum allocation standards, Wi-Fi is largely restricted to a small number of channels.
To overcome this challenge, the 802.11 protocol specifies a set of rules which allows only one device in the vicinity to transmit on a channel at the same time. 
This poses yet another challenge of distributing channels in a way that as few devices on the same channels as possible are close to one another. 

In short, while the physical layer (PHY) traditionally has been upgraded to meet the new demands in bitrate (e.g. fiber to the home, 
and MIMO in 802.11ac) Wireless LAN connections struggle under the heavy impact of RF-intereference which can not be solved by increasing
the physical datarate capacity. 

\section{Motivation}
Wireless LAN is deployed in almost all corporate buildings and residencies in the western world and increasingly also in the rest of the world.
The use of these networks used to be limited to laptops that generated small amounts of data, but now the range of devices includes
smart-phones, network storage devices, even in some cases servers. The deployment of Wi-Fi and the infrastructure has not 
changed much over the years to match the new demands and increased traffic, and in most places coverage is still the main concern, while
QoS comes second. 

Customers who are subscribed to high data rate service level agreements often find they can only receive a comparable data rate over wired LAN.
When the Wi-Fi network in their home is constantly underperforming, it is not unusual a customer to upgrade the data rate of their agreement. If interfering networks
were the perpatrators, usually increasing the bandwith will have no effect. This leads to customers being largely frustrated with their Internet service providers,
even though it is the underlying technology, and not the ISP which is at fault. A customer service representative or a tech-savvy consumer may manually
switch the operating channel for a wireless access point. If this has no effect a customer might be encouraged to get a router which can transmit a more powerful signal.
While this might give a short relief for the customer who was resourceful enough to deal with the problem, it in turn may trigger a chain reaction of even stronger
interference levels for the rest of the inhabitants in the surrounding area. 

There has been done research on how centralized controllers can benefit the deployment
of large enterprise networks (\cite{Murty}, \cite{Murty2} and \cite{Suresh}), but in this thesis we will address the emerging issue of deployment of Wi-Fi
in residential areas and how routers and access points could organize themselves to allow distributed control and coopeation on channel allocation, where centralized controllers
are impractical or impossible.

\section{Problem definition}
There are 3 non-overlapping channels on the 2.4GHz spectrum that 802.11 Wireless LANs uses. One of the common ways of selecting a channel
is done by letting an access point sense which channel has the lowest interference levels, also called least congested channel search.
When channels are selected in a selfish manner, where the only available information about the surrounding networks are obtained via
subjective radio readings, it is highly unlikely that the distribution of channels in a confined area (e.g. an apartment block) becomes optimal.  
Ideally all the access points would be configured so that the channel distribution in this area lead to as few equal channels as possible
being adjacent to one another. While this boils down to a NP-complete graph-coloring problem, it is not the focus of this thesis.
The main challenge to be considered is the clustering problem of identifying access points that heavily impacts eachother, and consider different approaches to how 
they can form groups in a distributed manner that seeks to intuitively identify nodes that lie withing the same confined areas, as apartment buildings
dense residential areas, etc. 

\section{Method}
To be able to see if we can form groups in a way that creates clusters of nearby nodes we need to get some data to perform
calculation on. We will both create syntethic data and use real world location data of access points to evaluate the performance of the group creations. 
Then we will take a look at possible solutions to let access points communicate with each other without any manual bootstrapping or 
previous association. Then we will consider the problems and challenges that has to be overcome in the process of creating and deploying 
an architecture as suggested in the thesis. 

    %We will take a look on earlier algorithms in the research of finding a better way to allocate channels in 802.11. 
    %Then we can do a number of assumptions to be able to propose an algorithm for group creation amongst unorganized access points,
    %and then evaluate the algorithm by computing groups and clusters of access points.  




%\section{Channel allocation} 
%To deal with the problem of channel allocation we will think of an AP as a vertex in a graph. When an AP scans its radio
%it can hear the strength of all nearby wireless networks measured in dBm (decibel milliwatts). This decibel value will be
%the value of the edge between one AP to another. With a graph expressing the wireless network topology, the problem
%of optimally distributing channels between APs boils down to a graph coloring problem. The number of colors in the color problem,
   %represents the number of non-overlapping channels in 802.11. Exactly how an algorithm can be designed to optimally distribute channels within the
   %interfering topology is out of the scope of this thesis. However we can define some invariants that has to be true
   %for such an algorithm to work:
   %\begin{enumerate} 
   %\item All APs has to run the same algorithm
   %\item All APs must run the algorithm on the same connected group
   %\item Because of the complexity of the problem the algorthm must solve, the number of APs in the connected group can not be too big
   %\end{enumerate}
%
   %Point 1 is trivial to solve or mitigate, as only APs running the algorithm will actively participate in the channel selection. A simple way to make sure that the
   %same algorithm is used, is by having a software version that is consistently checked with the other APs in the connected group.
%
   %Point 2 and point 3 is will be the main focus of the rest of the master thesis, as these are not so easily solved.
%
   %We can define a wireless topology graph as a set of wireless APs that are grouped together and share information about their neighbours and interference levels.
   %This set is what will now on be referred to as a \textit{connected group.} All members of the connected group will be considered when running the channel assignment algorithm.
   %For the connected group to have an actual impact on the quality of a network connection, it has to consist of nodes that normally disturbs each other substantially.
%
   %An ideal example of a connected group is an apartment builiding. The channel allocation protocol lets APs share information about who-disturbs-who the most in the building.
   %Then each AP can run the channel allocation algorithm. Because they run it on the same graph, every AP will find the same optimal channel distribution throughout the building,
   %and then switch to the correct channel. 
%
   %Even though an apartment building is most likely an optimal delimination of a connected group, in reality creating such a group is a bigger challenge. As the whole channel allocation
   %protocol is based on decentralized peer-to-peer technology, and no centralized server with access to demographical and geographical divisions exists, the protocol will
   %have to discover suitable connected groups on its own. Moreover, when the group is created the protocol will have to replicate data so that
   %all participants of the group has all the data required to perform channel allocation. It will also need a way to make sure that the image of the current group
   %is consistent within all APs in the connected group. 


