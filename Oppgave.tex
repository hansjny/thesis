\documentclass[11pt,a4paper,UKenglish]{report}
\usepackage[utf8]{inputenc}
\usepackage[T1]{fontenc,url}
\usepackage{babel,csquotes,newcent,textcomp,duomasterforside}
\usepackage[backend=biber,
	    style=numeric,
	    citestyle=numeric]{biblatex}

\addbibresource{references.bib}

\title{Group creation in a collaborative channel allocation scheme}
\subtitle{Subtitle}
\author{Hans Jørgen Furre Nygårdshaug}

\begin{document}
\duoforside[dept={Institutt for informatikk},
  program={Informatikk: programmering og nettverk},
  long]

\section{Disposition} 
\subsection {Introduction part}
\begin{itemize} 
	\item Introduction to the wifi interference problem
	\item Taking a step back and looking at other attempts on solving the problem
	\item Begin presenting Torleiv and Magnus work, the idea, and maybe the p2p protocol.
	\item End of by showing that there is a problem with creating, limiting and updating groups.
\end{itemize}

\subsection {Main thesis part}
\subsubsection{The problem of data replicaton} 
\begin{itemize} 
	\item The problem.
	\item Possible solutions? References.
	\item Complexity, out of scope for thesis. Assume problem is solved.
\end{itemize} 
\subsubsection{The algorithm itself} 
\begin{itemize} 
	\item Elaborating on the problem, introducing the first algorithm suggestion.
	\item Explain simulation data creation with stochastic uniform distribution.
	\item Show how the group creation algorith was created, design decisions (iterations etc).
	\item Results with visualizations through the visualization tool. 
	\item Evaluate results and consider improvements. How will this work in the wild?
	\item Introduce SSB data, the data format and why it is relevant. How is the tool made.
	\item Same procedure with result visualization and result evaluation. Do we still need improvements?
	\item Introducing Wigle as data source. Show results on map?
\end{itemize}

\subsection {Concluding part}
\begin{itemize} 
	\item Have we created a meaningful algorithm that can be implemented in hardware?
\end{itemize}

\clearpage


\section{Channel assignment} 
To deal with the problem of channel assignment we will think of an access point as a vertex in a graph. When an access point scans its radio,
it can see the strength of all nearby wireless networks measured in dBm (decibel milliwatts). This decibel value will be
the value of the edge between one access point to another. With such a graph expressing a wireless network topology, it boils down
to a graph coloring problem, where the number of colors represents the number of non-overlapping channels. Exactly how an algorithm can be designed
to optimally distribute channels within the interfering topology, is out of the scope of this thesis. However we can define some invariants that has to be true
for such an algorithm to work:
\begin{enumerate} 
	\item All access points has to run the same algorithm
	\item All access points must run the algorithm on the same wireless network topology graph
	\item Because of the complexity of the problem the number of access points in a network graph can not be too big
\end{enumerate}


As not all access points in the world
is relevant for the algorithm, and it can only perform well with a certain number of nodes, the system has to have a way to subdivide
the world in groups. Ideally a group is contained within a geographical location where a certain number of nodes is aggregated. If the
number of nodes in the group exceeds what the channel assignment algorithm can handle, the group has to be split into one or more new groups. 

\section{The Group Algorithm} 

\section{Simulation data}
Primarily, before beginning to implement and test the group creation algorithm, the task
is to create usable data to perform testing on. . Additionally, each node should have a neighbour list,
containing the respective interference measured in $-dBi$ for each neighbour. In addition,
the following parameters should be variable depending on each test scenario:

\begin{itemize}
\item Map size (width of x- and y-axis).
\item Number of nodes
\item Minimum distance between nodes (in meters)
\item Minimum measured $-dBi$ for a neighbour to consider it interfering
\end{itemize}

The program that generates the data is written in python, and can export the topology data to
JSON-format so it can be visualized or used by other applications.

The interference levels between access points is calculated by iterating through each node.
For each node $N$ we record its x and y position, and then start a second iteration through the nodes.
For each node in the second iteration $n$ we calculate the distance $d$ in
meters between $N$ and $n$ using Euclidean distance.



\printbibliography
\end{document}
