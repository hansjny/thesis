\chapter{Introduction}
- Expansion of WiFI
- Densely populated Areas
- Increased usage
- Expectations of universal wifi
- Congested networks because of intereference

\section{Problem definition}
The problem is that devices operating on the same channel in the same area interferes with one another.

\section{Motivation}
Wifi is deployed in almost all residencies, and the problem is especially prominent in densely populated areas. 

\section{Method}
Defining groups for running a channel allocation algorithm on

\section{Channel allocation} 
To deal with the problem of channel allocation we will think of an AP as a vertex in a graph. When an AP scans its radio
it can hear the strength of all nearby wireless networks measured in dBm (decibel milliwatts). This decibel value will be
the value of the edge between one AP to another. With a graph expressing the wireless network topology, the problem
of optimally distributing channels between APs boils down to a graph coloring problem. The number of colors in the color problem,
   represents the number of non-overlapping channels in 802.11. Exactly how an algorithm can be designed to optimally distribute channels within the
   interfering topology is out of the scope of this thesis. However we can define some invariants that has to be true
   for such an algorithm to work:
   \begin{enumerate} 
   \item All APs has to run the same algorithm
   \item All APs must run the algorithm on the same connected group
   \item Because of the complexity of the problem the algorthm must solve, the number of APs in the connected group can not be too big
   \end{enumerate}

   Point 1 is trivial to solve or mitigate, as only APs running the algorithm will actively participate in the channel selection. A simple way to make sure that the
   same algorithm is used, is by having a software version that is consistently checked with the other APs in the connected group.

   Point 2 and point 3 is will be the main focus of the rest of the master thesis, as these are not so easily solved.

   We can define a wireless topology graph as a set of wireless APs that are grouped together and share information about their neighbours and interference levels.
   This set is what will now on be referred to as a \textit{connected group.} All members of the connected group will be considered when running the channel assignment algorithm.
   For the connected group to have an actual impact on the quality of a network connection, it has to consist of nodes that normally disturbs each other substantially.

   An ideal example of a connected group is an apartment builiding. The channel allocation protocol lets APs share information about who-disturbs-who the most in the building.
   Then each AP can run the channel allocation algorithm. Because they run it on the same graph, every AP will find the same optimal channel distribution throughout the building,
   and then switch to the correct channel. 

   Even though an apartment building is most likely an optimal delimination of a connected group, in reality creating such a group is a bigger challenge. As the whole channel allocation
   protocol is based on decentralized peer-to-peer technology, and no centralized server with access to demographical and geographical divisions exists, the protocol will
   have to discover suitable connected groups on its own. Moreover, when the group is created the protocol will have to replicate data so that
   all participants of the group has all the data required to perform channel allocation. It will also need a way to make sure that the image of the current group
   is consistent within all APs in the connected group. 


