\chapter{Introduction}
The presence of Wi-Fi is ever expanding, and by 2020 it is esitmated that we have 20 billion connected devices \cite{Gartner}.
According to the Wi-Fi Alliance, households alone will be home to over 10 billion devices that are able to connect to
the home router \cite{wifialliance}. 
Even while introducing more and more devices to a home in the shape of smart TVs, lights, and audio systems, at the same time
it is expected that the WiFi connection will deliver optimal speeds to meet our time's expectations of universal wireless access. 
Not only is the number of devices growing, but our traffic patterns has changed from bursty transmission of text, images and occasional
video footage, to continous streams of high bitrate video and audio. While fiber connections are growing in popularity to meet the 
traffic demand, WiFi connections struggle under the heavy impact of neigbouring traffic intereference in densely populated areas. 

\section{Problem definition}
There are 3 non-overlapping channels on the 2.4GHz spectrum that 802.11 WiFi uses. The usual way of selecting a channel is done by
letting an access point sense which channel has the least amount of interference. This can lead to excessive
channel switching and an unstable environment with chain reactions. Ideally all access points would be configured so
the channel allocation is optimal for an entire area. This can be realized with a centralized controller,
but in residential WiFi networks there are by default no centralized controller. We will explore how wireless access points can
organize themselves in confined groups that collaborate on selecting an optimal channel distribution for the entire group
to maximize the efficiency of wireless networks.

\section{Motivation}
WiFi is deployed in almost all residencies in the western world, which makes the problem of interfering channels
especially prominent in densely populated areas. When customers are subscribed to high data rate service level agreements,
but can only receive a fraction of the announced data rate over wifi and is adviced to switch to cable, there is a need for improvement. 

\section{Method}
We will take a look on what has been done earlier in the research of finding a better way to allocate channels in 802.11. 
Then we can do a number of assumptions to be able to propose an algorithm for group creation amongst unorganized access points,
and then evaluate the algorithm by computing groups and clusters of access points.  

%\section{Channel allocation} 
%To deal with the problem of channel allocation we will think of an AP as a vertex in a graph. When an AP scans its radio
%it can hear the strength of all nearby wireless networks measured in dBm (decibel milliwatts). This decibel value will be
%the value of the edge between one AP to another. With a graph expressing the wireless network topology, the problem
%of optimally distributing channels between APs boils down to a graph coloring problem. The number of colors in the color problem,
   %represents the number of non-overlapping channels in 802.11. Exactly how an algorithm can be designed to optimally distribute channels within the
   %interfering topology is out of the scope of this thesis. However we can define some invariants that has to be true
   %for such an algorithm to work:
   %\begin{enumerate} 
   %\item All APs has to run the same algorithm
   %\item All APs must run the algorithm on the same connected group
   %\item Because of the complexity of the problem the algorthm must solve, the number of APs in the connected group can not be too big
   %\end{enumerate}
%
   %Point 1 is trivial to solve or mitigate, as only APs running the algorithm will actively participate in the channel selection. A simple way to make sure that the
   %same algorithm is used, is by having a software version that is consistently checked with the other APs in the connected group.
%
   %Point 2 and point 3 is will be the main focus of the rest of the master thesis, as these are not so easily solved.
%
   %We can define a wireless topology graph as a set of wireless APs that are grouped together and share information about their neighbours and interference levels.
   %This set is what will now on be referred to as a \textit{connected group.} All members of the connected group will be considered when running the channel assignment algorithm.
   %For the connected group to have an actual impact on the quality of a network connection, it has to consist of nodes that normally disturbs each other substantially.
%
   %An ideal example of a connected group is an apartment builiding. The channel allocation protocol lets APs share information about who-disturbs-who the most in the building.
   %Then each AP can run the channel allocation algorithm. Because they run it on the same graph, every AP will find the same optimal channel distribution throughout the building,
   %and then switch to the correct channel. 
%
   %Even though an apartment building is most likely an optimal delimination of a connected group, in reality creating such a group is a bigger challenge. As the whole channel allocation
   %protocol is based on decentralized peer-to-peer technology, and no centralized server with access to demographical and geographical divisions exists, the protocol will
   %have to discover suitable connected groups on its own. Moreover, when the group is created the protocol will have to replicate data so that
   %all participants of the group has all the data required to perform channel allocation. It will also need a way to make sure that the image of the current group
   %is consistent within all APs in the connected group. 


