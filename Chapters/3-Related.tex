\chapter{Related work}
This chapter is dedicated to give a brief account of works that addresses the issue of controlling and managing 802.11 wireless access points. It is constrained
to technologies that in some way provide channel allocation or other ways to reduce RF-interference using more information than what is available at each individual access point.

\section{Centralized Solutions}

\subsection{Cisco RRM}
CISCO offers a RRM (Radio Resource Management) solution directed at enterprise networks \cite{ciscoRRM}.
Their architecture consists of one or more Wireless Lan Controllers (WLCs), which performs the responsibilities of regular APs, 
but also performs RRM, through grouping, leader election, dynamic channel assignment and transmit power control. Regular APs that 
are not WLCs, tunnel all their packets through the WLC unit they are registered with \cite{cisco_2009}. There are many finesses and details to the Cisco RRM, here we will only present the
major concepts.

\subsubsection{RF-Grouping}
In CISCO RRM, the RF-grouping stage is the identification of controllers and APs that are under the administrative control of the same RRM. 
To identify the RF-group an RF-group name is used. The name itself is an ascii string, and the APs obtain it from the WLC they are connected to. The WLCs have to be preconfigured with this group name.

All APs broadcasts their RF-group name along with the IP-address of their controller.
Any other AP sharing the same RF-group name reports the incoming broadcast message to the controller, and then rebroadcasts the packet similar to the flooding routing mechanism.
After all APs has done this, the RF-group can identify all APs that can physically hear each other.
This is called the neighbour discovery stage, and APs within physical hearing of -80dBi and above makes up what Cisco calls an RF-neighbourhood.
If applicable the RF-groups perform leader election to decide which controller becomes the RF Group leader, but a preconfigured leader can also be decided.
The RF group leader has the responsibility of running the relevant channel allocation algorithms and adjusting the radio power level of the APs. 

\subsubsection{Dynamic Channel Assignment}
To perform Dynamic Channel Assignment, the RF-leader has the neighbour lists for all access points in 
its RF-group. All APs track a set of metrics which tells something about its transmission quality. The metrics are: 
\begin{itemize}
	\item Co-channel interference from RF-group
	\item Co-channel interference from rogue APs (not a part of the group)
	\item Background noise that does not originate from 802.11
	\item Channel load measured at the physical layer
	\item Sensitivity to change (user selected variable to indicate how fast the network should respond to bad quality)
\end{itemize}
These metrics together end up in a unified metric called the Cost Metric, and the Cost Metric is then reported to the RF-leader. 
The DCA algorithm looks at the AP with the lowest cost metric first, meaning the AP with the worst QoS. The transmission channel can be changed for
this AP, and all 1-hop neighbours of this AP, but no AP further away than 1-hop. This is to prevent impacting APs throughout the RF-group.
That process is repeated, computing several channel plans, and at last, all the channel plans are evaluated through a cost function
to see if the Cost Metric has improved for the AP, without degrading the Cost Metric for neighbouring nodes. If an improvement is seen
then the channels are changed, and the algorithm moves on to the next AP on the list of low Cost Metrics.  

\subsubsection{Assessment}
It is valuable to understand the major mechanisms of Cisco RRM, as it is among the leading technologies for administrating larger amount of wireless access points. 
The RF-grouping stage with the neighbour discovery protocol are mechanisms that may be adaptable, or at least an inspiration for a future distributed protocol. The reason 
the Cisco RRM would not work well as it is in a chaotic, residential deployment may be quite obvious: the need for preconfigured WLCs and proprietary hardware does not fit well with the nomadic 
routers today, where house and apartment owners bring their own routers and access points. 



\subsection{DenseAP}
DenseAP described in \cite{Murty2} aims to restructure the infrastructure of enterprise networks.  The two fundamental
changes they suggest is to deploy APs a lot denser (hence the name), and moving the task of associating clients
with APs to a centralized controller. The reasoning behind the dense deployment of APs is that signals diminishes
quickly in an indoor environment, and ideally a client should always be associated with an AP in the close vicinity. The argument 
they use for moving the association decision away from the client and over to a central controller, is that a client can
only use signal strength as the metric deciding which AP to associate with. This is emphasized as
suboptimal in conferences and meeting room environments, where many clients seek to associate with an AP at the same time.
If all clients pick the same AP it will naturally reduce the throughput on the medium because of the CSMA/CA protocol. 
Their infrastructure consists of DenseAP access points (DAPs) and DenseAP Controllers (DCs). The DAPs sends periodic
reports to the DC, which contains information as RSSI (Received Signal Strength Indication) measures, co-channel interference, and associated clients. Based on this
information the DC decides which DAP each client should associate with, and also which channel each DAP should transmit
on. 

\subsubsection{Assessment}
If we are to compare this solution with the Cisco RRM, the only thing in common is that this technology is also designed for enterprise networks to increase QoS over Wi-Fi. There
are very few other similarities, and while Cisco strives to optimize the channel plan and control of the transmission power of the antennas, DenseAP suggests having more APs, and control client association. This technology focuses less on optimizing the channel plan, and more on providing many different channels in the same area.  


\subsection{HiveOS}
HiveOs \cite{Aerohive}, developed by Aerohive Networks offers distributed protocols and mechanisms to improve Wireless LANs in enterprise networks. The APs 
in the network are called HiveAPs, and they offer services such as
\begin{itemize}
	\item Band steering: if an device can operate on the 5Ghz band, it will be forced to connect to the 5Ghz network to optimize the utilization of the radio spectrum. 
	\item Load balancing: all HiveAPs have real-time information about how clients performs. If a client tries to associate with a new HiveAP, it will only be accepted
				if the new HiveAP has a low enough load to handle more clients. It would also know if other neighbouring APs are better suited to handle the load of the new client.
				This is achieved by withholding probe responses.

	\item Channel allocation: by using the Aerohive Channel Selection Protocol, HiveAPs tries to select the channel with the lowest co-channel interference. Their channel selection protocol uses 5 measurements, two static and 3 dynamic.
	The first static measurements is the number of nearby APs who are operating on the same channel. The more APs there are the higher the penalty. The penalty per AP diminishes as the number of APs increases, this
		is because the first one(s) are the most critical. The other static cost factor is what power level can be transmitted at the given channel, as this may differ on some 5GHz non-overlapping channels. The dynamic measurements are CRC-error rate, channel utilization,
	and the utilization of overlapping APs. All of these dynamic factors can penalize a channel with 0 to 3.5%. 

\subsection{}
\end{itemize}

\section{ResFi}
\subsubsection{Description}
ResFi \cite{resfi} is an protocol that aims to enable self-organized management in residential deployments of wireless LAN through communication between access points. Works such as DenseAP \cite{Murty2}, Cisco RRM\cite{ciscoRRM} and Aerohive \cite{Aerohive} are all directed toward enterprise networks.

ResFi assumes that all access points have two interfaces, one connected to a wired backhaul (like the Internet), and another
802.11 compatible wireless interface. ResFi communicates with the device that interfaces with the antenna, which in most residential homes will be a router. This control the operating channel and the transmission power levels of the antenna. In short ResFi enables communication between access points under different basic service sets, without imposing a central controller on the access points or
requiring them to be a part of an extended service set. If two access points are running ResFi and are within physical hearing range of each other, then ResFi 
establishes a secure communication channel between these APs over the wired backhaul. ResFi enables all of this without doing any modifications to hardware and drivers (like modifying standards or requiring proprietary equipment). 

The services provided by ResFi are 1-hop unicast communication and n-hop broadcast messaging between access points. These services can be valuable for an
implementation of a distributed clustering algorithm, where access points
may need to perform message passing between each other to coordinate and form groups. 

ResFi uses a modified version of hostapd (host access point daemon), a user space application, which means any
protocols that runs on top of ResFi does not require modification to kernel modules or drivers. This is unquestionably a huge benefit for the development and deployment of
new services that benefits from communication between access points, eliminating the need for routers with a specific firmware to be compatible. 
We will take a closer look at this technology in chapter \ref{chap:proto}, where we more closely discuss how it can be used to enable the
communication channel between access points to run a distributed clustering algorithm. 


\section{Channel allocation using DSATUR} 
This section is dedicated to previous work done using the DSATUR heuristic.

DSATUR (from degree of saturation) is a heuristic created by Daniel Brélaz \cite{Brelaz} to find solutions to the NP-hard problem of coloring the vertices of a graph so that no adjacent vertices share the same color.  With a network topology mapped out, this heuristic can be used to optimize the distribution of channels. This is done by treating access points as vertices in a graph, and different channels as colors. Vertices connected by an edge indicates that the two access points can physically hear each other, weight of the edge being the strength of which they can observe each other.  An
important property of DSATUR its deterministic nature. Meaning that if different access points have an equal graph, then all access points will compute the same channel distribution.

Channel allocation schemes relying on the DSATUR algorithm has been proposed before. In 2004 a paper was published, called
"Automatic channel allocation for small wireless local area networks using graph colouring algorithm approach" \cite{mahonen}.

The paper suggest building the graph by using flooding, and its done in the following way:
\begin{itemize}
	\item When an access points boots up, it waits for 802.11 beacon messages from other APs to discover its neighbours. A neighbour list is built using their MAC-addresses. 
	\item After all neighbours have been discovered, it broadcasts its list of neighbouring MAC-addresses to all neighbours.  
	\item When an AP receives a broadcasted neighbour list, it rebroadcasts the list to all its own neighbours. The information from the list is used to build the network graph. 
\end{itemize}

To prevent eternal propagation of the same message, all nodes only rebroadcast a list one time, and to keep the lists up-to-date, each node has to broadcast new
lists periodically. When the protocol converges, all access points in the area should be a part of the graph, and the graph should be equal for all access points. The DSATUR algorithm can then be used to compute the channel distribution.

This protocol does in fact identify a cluster of access points. However, the flooding mechanism used to identify clusters
would most likely not work well today. In 2004 the low density of access points meant that the protocol would converge quickly, but in some areas today it might never converge because
of the vast amount of access points. If it did converge, the amount of nodes in the graph could potentially be way beyond what is plausible to solve when the problem is NP-hard and with the computational power of an access point. 

What it lacks is a way to constrain the size of the graph, and a method to distinguish more disturbing nodes from nodes that causes little interference.

%
\section{Distributed clustering in ad-hoc networks}
This section is dedicated to briefly address distributed clustering algorithms used in ad-hoc networks.

In ad-hoc networks all access points in the network has to implement a routing mechanism with the goal of establishing end-to-end communication 
between APs without a wired backhaul network. As all APs has to act as routers, they have the roles of sending routing messages, as well as relaying data messages
going to and from neighbouring nodes. If the ad-hoc network structure is flat, routing does not scale well. Thus, just as the Internet, ad-hoc networks benefits
from creating a hierarchical topology. Algorithms such as DCA, DMAC  and GDMAC \cite{Basagni} seeks to partition network topologies into clusters to form a hierarchical network structure.
Common for these algorithms is that information flow between clusters go through so called Cluster Heads (CH). These clustering algorithms focuses on finding CHs that will hold routing information and topology information. The purpose is that every neighbouring node can route their inter-cluster traffic through the CH node, to simplify addressing and reduce routing traffic \cite{Mamalis}.

According to \cite{Basagni} , the following properties has to be satisfied for any ad-hoc clustering algorithm:
\begin{enumerate}
	\item Every node in the topology that is not a CH, has a CH as a neighbour.
	\item All nodes select the neighbouring CH which has the highest weight
	\item CHs can not be neighbours
\end{enumerate}

These clustering algorithms are mainly concerned with identifying 1-hop cluster heads. Hence, even though the naming is similar, the goal of these clustering algorithms is
different from the purposes of the distributed clustering algorithm we develop in this thesis.
%They seek to in some way minimize traffic, power, 
%In our clusters should focus on minimizing the interference between clusters, as well as constraining the cluster size.  



%SCIFI \cite{SCIFI} is a centralized channel allocation protocol for infrastructure Wireless LANs that improves the traditional graph-coloring algorithm DSATUR. 
%While the paper shows that their central coordinator in fact improves the throughput compared to Wireless LANs that are not configured by SCIFI, the it is an algorithm for setting a channel in a preconfigured administrative domain. So, why is still mention here? Well, even though it does not deal with how the administrative domains
%or AP clusters are defined, if the method of creating clusters of collaborating routers proposed in this thesis has merit,
%SCIFI could plausibly be a supporting technology to compute channel distribution. Meaning one of our groups would act as the administrative domain required by SCIFI. 


