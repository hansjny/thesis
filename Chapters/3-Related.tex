\chapter{Related work}

\section{Brief survey of related work}
\subsection{Cisco RRM}
CISCO offers a solution directed at enterprise networks \cite{ciscoRRM}, where implementing and managing a centralized
controller is a more managable task than in residential networks.
Their architecture consists of one or more Wireless Lan Controllers (WLCs).
The WLCs creates a group name which all APs part of the same RF-group is aware of. All APs broadcasts their RF-group name, along
with the IP-address of their controller. Any other AP sharing the same RF-group name reports the incoming broadcast message to the controller, and then rebroadcasts the packet.  This way the controller
becomes aware of all APs that can hear each other, similar to the flooding routing mechanism,
and they can form logical groups based on this information.
If applicable the RF-groups perform leader election to decide which controller becomes the RF Group leader, but there
can also a preconfigured leader. The RF group leader has the responsibility of running the relevant channel allocation
algorithms and ajusting the radio power level of the APs. 

\subsection{DenseAP}
DenseAP described in \cite{Murty2} aims to restructure the infrastructure of enterprise networks.  The two fundamental
changes they suggest is to deploy APs a lot denser (hence the name), and moving the task of associating clients
with APs to a centralized controller. The reasoning behind the dense deployment of APs is that signals diminishes
quickly in an indoor environment, and ideally an AP can always be associated with an AP in the close vicinity. The argument 
they use for moving the association decision away from the client and over to a central controller, is that a client
only uses signal strength as the metric to decide which to associative with. They emphasize that this can be
suboptimal in conferences and meeting room environments, where many clients seek to associate with an AP at the same time.
If all clients pick the same AP it also means all clients will transmit on the same channel,
and RF-interference can reduce the throughput on the medium. 
Their infrastructure consists of DenseAP access points (DAPs) and DenseAP Controllers (DCs). The DAPs sends periodic
reports to the DC, which contains information as RSSI measures, channel interference, and associated clients. Based on this
information the DC decides which DAP each client should associate with, and also which channel each DAP should transmit
on. 


\subsection{HiveOS}
HiveOs \cite{Aerohive}, developed by Aerohive Networks offers distributed protocols and mechanisms to improve Wireless LANs in enterprise networks. The APs 
in the network are called HiveAPs, and they offer services such as
\begin{itemize}
	\item Band steering: if an device can operate on the 5Ghz band, it will be forced to connect to the 5Ghz network to optimize the utilization of the radio spectrum. 
	\item Load balancing: all HiveAPs have real-time information about how clients performs. If a client tries to associate with a new HiveAP, it will only be accepted
				if the new HiveAP has a low enough load to handle more clients. It would also know if other neighbouring APs are better suited to handle the load of the new client.
				This is achieved by witholding probe responses. 
	\item Channel allocation: by using the Aerohive Channel Selection Protocol, HiveAPs tries to select the channel with the lowest co-channel interference. Their channel selection protocol uses 5 measurements, two static and 3 dynamic.
	The 2 static measurements are the number of nearby APs are operating on the same channel. The APs there are the higher the penalty, but the penalty per AP reduces as the number of APs increases, as the first one(s) are the most
	critical. The other static cost factor is what power level can be transmitted at the given channel, as this may differ on some 5GHz non-overlapping channels. The dynamic measurements are CRC-error rate, channel utlization,
	and the utilization of overlapping APs. All of these dynamic factors can penalize a channel with 0 to 3.5%. 
\end{itemize}



\subsection{ResFi}
\subsection{Distributed Clusteering}
DCA and DMAC

\subsection{SCIFI} SCIFI \cite{SCIFI} is a centralized channel allocation protocol for infrastructure Wireless LANs that improves the traditional graph-coloring algorith DSATUR. 
		And while it shows that their central coordinator in fact improves the throughput compared to Wireless LANs that are not configured by SCIFI, the algorithm is just an algorithm for setting a channel in a preconfigured adminstrative domain. It does not deal with how the administrative domains (or collaborative groups as we call it) are defined. 


