\documentclass[12pt, a4paper, UKenglish, final]{report}
\usepackage[utf8]{inputenc}
\usepackage{amsmath}
%\usepackage{mathtools,amssymb}
\usepackage{algorithm2e}
%\usepackage[noend]{algpseudocode}
\usepackage{babel,csquotes}
\usepackage{float}
\usepackage{ifimasterforside}
\usepackage{listings}
\usepackage{wrapfig}
\usepackage{graphicx}
\usepackage{fancyhdr}
\usepackage[toc,page]{appendix}
\usepackage[backend=biber,
	    style=numeric,
	    citestyle=numeric]{biblatex}

\addbibresource{references.bib}


\usepackage{xcolor}
\usepackage{wrapfig}
\usepackage{caption}
\usepackage{subcaption}
\pagestyle{fancy}
\fancyhead{}
\fancyhead[RO,LE]{\leftmark }
\fancyfoot{}
\fancyfoot[CO,RE]{\thepage}

\makeatletter

\newcommand\frontmatter{%
	\cleardoublepage
	%\@mainmatterfalse
  \pagenumbering{roman}}

\newcommand\mainmatter{%
	\cleardoublepage
	% \@mainmattertrue
  \pagenumbering{arabic}}

\newcommand\backmatter{%
	\if@openright
    \cleardoublepage
  \else
    \clearpage
  \fi
  % \@mainmatterfalse
   }

\DeclareFixedFont{\ttb}{T1}{txtt}{bx}{n}{11} % for bold
\DeclareFixedFont{\ttm}{T1}{txtt}{m}{n}{11}  % for normal

% Custom colors
\usepackage{color}
\definecolor{deepblue}{rgb}{0,0,0.5}
\definecolor{deepred}{rgb}{0.6,0,0}
\definecolor{deepgreen}{rgb}{0,0.5,0}

\usepackage{listings}

% Python style for highlighting
\newcommand\pythonstyle{\lstset{
language=Python,
stepnumber=1, 
tabsize=2, 
breaklines=true,
basicstyle=\scriptsize\ttfamily,
otherkeywords={self},             % Add keywords here
keywordstyle=\scriptsize\ttfamily\color{deepblue},
%emph={MyClass,__init__},          % Custom highlighting
%emphstyle=\ttb\color{deepred},    % Custom highlighting style
%stringstyle=\color{deepgreen},
%frame=tb,                         % Any extra options here
%showstringspaces=false            % 
}}


% Python environment
\lstnewenvironment{python}[1][]
{
	\pythonstyle
\lstset{#1}
}
{}

% Python for external files
\newcommand\pythonexternal[2][]{{
	\pythonstyle
\lstinputlisting[#1]{#2}}}

% Python for inline
\newcommand\pythoninline[1]{{\pythonstyle\lstinline!#1!}}
\makeatother
\colorlet{punct}{red!60!black}
\definecolor{background}{HTML}{EEEEEE}
\definecolor{delim}{RGB}{20,105,176}
\colorlet{numb}{magenta!60!black}

\lstdefinelanguage{json}{
	basicstyle=\scriptsize\ttfamily,
    numbers=left,
    numberstyle=\scriptsize,
    stepnumber=1,
    numbersep=2pt,
    showstringspaces=false,
    breaklines=true,
    backgroundcolor=\color{background},
    literate=
     *{0}{{{\color{numb}0}}}{1}
      {1}{{{\color{numb}1}}}{1}
      {2}{{{\color{numb}2}}}{1}
      {3}{{{\color{numb}3}}}{1}
      {4}{{{\color{numb}4}}}{1}
      {5}{{{\color{numb}5}}}{1}
      {6}{{{\color{numb}6}}}{1}
      {7}{{{\color{numb}7}}}{1}
      {8}{{{\color{numb}8}}}{1}
      {9}{{{\color{numb}9}}}{1}
      {:}{{{\color{punct}{:}}}}{1}
      {,}{{{\color{punct}{,}}}}{1}
      {\{}{{{\color{delim}{\{}}}}{1}
      {\}}{{{\color{delim}{\}}}}}{1}
      {[}{{{\color{delim}{[}}}}{1}
      {]}{{{\color{delim}{]}}}}{1},
}

\title{Group creation in a collaborative P2P channel allocation protocol}
\subtitle{Identifying connected groups of access points}
\author{Hans Jørgen Furre Nygårdshaug}

\begin{document}
\ififorside
\maketitle{}
\frontmatter

\section*{Abstract}
\clearpage
%\section*{List of acronyms}
%\begin{tabular}{ c c }
	%\textbf{AP} & Access Point\\ 
	%\textbf{CSMA/CA} & Carrier Sense Multiple Access / Collision Avoidance\\
	%\textbf{MAC} & Medium Access Control\\
	%\textbf{CCA} & Clear Channel Assessment\\
	%\textbf{MAC} & Mandatory Access Control\\
	%\textbf{PHY} & Physical Layer\\
	%\textbf{PLCP} & Physical Layer Convergence Procedure\\
	%\textbf{PPDU} & PLCP Protocol Data Unit\\
%\end{tabular}
\tableofcontents
\listoffigures

\mainmatter
\chapter{Introduction}
The amount of Internet connected devices has for the past few years been rapidly increasing, and is still increasing this day. The latest forecasts
esitmates that there will be about 27 billion devices connected to the Internet by 2021 \cite{CiscoVNI2017}, while by the end of the 2017 the number
connected devices will be close to 8.4 billion \cite{Gartner}. In 2020 households alone will be responsible for over 10 billion devices
able to wirelessly connect to the home router \cite{wifialliance}, and wireless traffic will account for 63 percent of all IP-traffic in the world \cite{CiscoVNI2017}. 
Traditional devices connected to the Internet like computers, phones, watches, smart TVs, audio systems, and lately also private network storage systems is responsible
for much of the traffic. However, as the era of Internet of Things (IoT) has rapidly descended upon us, increasingly also less obvious utilites are connected to the Internet.
These devices may span everything from lights and HVAC systems to coffee machines, fridges and even toasters. Whereas in the early 2000s it was common to own 
one or two MAC-addresses per person, at the date of writing it is not unusual to possess a two digit number of MAC-addresses.

But it is not only the numbers of devices that has changed. The consumers' expectations and demands are also being altered over time. 
While ubiquitous connectivity is a buzzword often heard in the context of future 5G networks, ubiquitous access is already expected in
modern households. The demand for Wi-Fi coverage extends through the entire home: the Internet radio in the garage, the video stream in the basement
couch and the gaming laptop in the bedroom. All demands coverage and expects mobility at the same time. Many of these devices also have something else in
common which reflects the change that has happened over the last few years: demand for high data rate. In 2016 about 73 percent of all IP traffic orginated from video
streams, and in 2021 this number is expected to increase all the way up to 82 percent \cite{CiscoVNI2017}. 

The new demand for continous streams of high bitrate video poses a challenge that can not simply be solved by providing larger bandwidth. First of all, video streams
have high Quality of Service (QoS) demands. Naturally, buffering of videos and/or reduction of video quality will negatively impact the experience of the service.
Secondly, it is not unusual that different video flows occur simultaneously in the same household. If such traffic is transmitted wirelessly
it results in an inherently busy transmission medium. Because of the international spectrum allocation standards, Wi-Fi is largely restricted to a small number of channels.
To overcome this challenge, the 802.11 protocol specifies a set of rules which allows only one device in the vicinity to transmit on a channel at the same time. 
This presents a largely unresolved challenge of mapping channels to access points in a way that there are only a few, or better yet no, devices on the same channels adjacent
to each other.

While the physical layer traditionally has been upgraded to meet the new demands in bitrate (e.g. fiber to the home, 
and MIMO in 802.11ac) Wireless LAN connections struggle under the heavy impact of radio frequency intereference which can not be solved by increasing
the physical datarate capacity.

There has been done research on how centralized controllers can benefit the deployment of large enterprise networks (\cite{Murty}, \cite{Murty2} and \cite{Suresh}),
but in this thesis we will address the emerging issue of deployment of Wi-Fi in residential areas. More specifically, we will consider possible clustering methods
to enable routers and access points to organize themselves in self managing groups. A synchronous, distributed group across access points in different service sets
would allow for planned, cooperative channel allocation. This would transform the channel allocation problem from finding a local optimal channel for each router,
to a problem of finding an optimal channel plan for the entire group.

\section{Motivation}
Wireless LAN, commonly referred to as Wi-Fi, is deployed in almost all corporate buildings and residencies in the modern world.
The use of these networks used to be limited to laptops or phones that generated small amounts of data, but now the range of devices includes
smart-phones, network storage devices, and IoT appliances. The deployment of Wi-Fi and the infrastructure has not 
changed much over the years to match the new demands and increased traffic. Actually, in most places coverage is still the main concern, while
QoS comes second. 

Customers subscribed to high data rate service level agreements often find they can only receive a comparable data rate over wired LAN.
When the Wi-Fi network in their home is constantly underperforming, it is not unusual a customer to upgrade the data rate of their agreement.
But as it happens, often interfering networks are the perpetrators, and an increase of bandwith will have no effect.
This can lead customers to frustration with their Internet service providers, even though it is the underlying technology and not the service provider which is at fault.

There exists a large amount of different algorithms that deal with channel allocation to prevent and limit interference. Some even consider the QoS observed by the client - not only the access points.
However, many efficient algorithms are deployed in centralized systems and does not focus on decentralized instances, such as residential networks.
A decentralized, distributed solution to optimize channel distribution would be helpful in apartment buildings and other residential zones where the density of wireless networks is high,
as it would not require the wireless networks to be under the administration of the same controller (e.g. an ISP). The access points could communicate and organize channel allocation
on their own. Of all channel allocation algorithms only a few addresses the issue of optimizing the channel distribution,
but the algorithms does not have a way to limit the amount of nodes to consider.  The main motivation behind this thesis is to explore ways to create a distributed group (clustering) scheme. Once a group is established, many problems can be solved with technologies previously only used in centralized systems.

\section{Problem definition}
There are 3 non-overlapping channels on the 2.4GHz spectrum utilized by 802.11 Wi-Fi. Today, one of the more common ways of selecting a channel
is done by letting an access point sense which channel has the lowest interference levels, also called least congested channel search.
When channels are selected in this selfish manner, where the only available information about the surrounding networks are obtained via
local radio observations, it is highly unlikely that the channel distribution in a confined area (e.g. an apartment block) becomes optimal.  

Ideally we want all access points to be configured so the channel distribution in an area leads to as few equal channels as possible
being next to one another. The task of optimizing channel distribution is essentially what is called a graph-coloring problem, where the problem is
finding a distribution where no adjacent nodes should operate on the same channels. This is NP-hard and can be solved with heuristics \cite{Brelaz}, but is not the focus of this thesis. 

Since the problem is NP-hard, it is easier to determine a good channel distribution if fewer access points are considered. In other words, if the groups are too large, 
the channel distribution problem becomes too complicated to solve. This bring us to the punch line of the problem definition: finding a way to confine access points to a high-impact group.
We have already established the motivation behind aiming for a decentralized solution - but to achieve this, access points have to find these groups on their own.
It is easy to intuitively say that, for instance, an apartment building should be one group. It is less obvious to answer how all access points in an apartment building
would identify the boundaries of the building and create a group on their own, entirely independent of an Internet service provide (and ideally also router brand).
It boils down to a distributed clustering problem, where no access point has a global view of the landscape of neighbouring routers. So we must find a way to make the clustering algorithm
able to be run in a way that clusters can be built relying only on information that can be obtained from radio scans of its membering access points.

\section{Method}
To be able to see if we can form groups in a way that creates clusters of nearby nodes we need to get some data to perform
calculation on. We will both create syntethic data and use real world location data of access points to evaluate the performance of the group creations. 
Then we will take a look at possible solutions to let access points communicate with each other without any manual bootstrapping or 
previous association. Then we will consider the problems and challenges that has to be overcome in the process of creating and deploying 
an architecture as suggested in the thesis. 

    %We will take a look on earlier algorithms in the research of finding a better way to allocate channels in 802.11. 
    %Then we can do a number of assumptions to be able to propose an algorithm for group creation amongst unorganized access points,
    %and then evaluate the algorithm by computing groups and clusters of access points.  




%\section{Channel allocation} 
%To deal with the problem of channel allocation we will think of an AP as a vertex in a graph. When an AP scans its radio
%it can hear the strength of all nearby wireless networks measured in dBm (decibel milliwatts). This decibel value will be
%the value of the edge between one AP to another. With a graph expressing the wireless network topology, the problem
%of optimally distributing channels between APs boils down to a graph coloring problem. The number of colors in the color problem,
   %represents the number of non-overlapping channels in 802.11. Exactly how an algorithm can be designed to optimally distribute channels within the
   %interfering topology is out of the scope of this thesis. However we can define some invariants that has to be true
   %for such an algorithm to work:
   %\begin{enumerate} 
   %\item All APs has to run the same algorithm
   %\item All APs must run the algorithm on the same connected group
   %\item Because of the complexity of the problem the algorthm must solve, the number of APs in the connected group can not be too big
   %\end{enumerate}
%
   %Point 1 is trivial to solve or mitigate, as only APs running the algorithm will actively participate in the channel selection. A simple way to make sure that the
   %same algorithm is used, is by having a software version that is consistently checked with the other APs in the connected group.
%
   %Point 2 and point 3 is will be the main focus of the rest of the master thesis, as these are not so easily solved.
%
   %We can define a wireless topology graph as a set of wireless APs that are grouped together and share information about their neighbours and interference levels.
   %This set is what will now on be referred to as a \textit{connected group.} All members of the connected group will be considered when running the channel assignment algorithm.
   %For the connected group to have an actual impact on the quality of a network connection, it has to consist of nodes that normally disturbs each other substantially.
%
   %An ideal example of a connected group is an apartment builiding. The channel allocation protocol lets APs share information about who-disturbs-who the most in the building.
   %Then each AP can run the channel allocation algorithm. Because they run it on the same graph, every AP will find the same optimal channel distribution throughout the building,
   %and then switch to the correct channel. 
%
   %Even though an apartment building is most likely an optimal delimination of a connected group, in reality creating such a group is a bigger challenge. As the whole channel allocation
   %protocol is based on decentralized peer-to-peer technology, and no centralized server with access to demographical and geographical divisions exists, the protocol will
   %have to discover suitable connected groups on its own. Moreover, when the group is created the protocol will have to replicate data so that
   %all participants of the group has all the data required to perform channel allocation. It will also need a way to make sure that the image of the current group
   %is consistent within all APs in the connected group. 



\chapter{Background}
In this chapter the relevant aspects of wireless technology and a selection of important
concepts from the 802.11 standard will be introduced.

\section{Basic radio challenges}
There are some challenges with wireless technologies that are harder to overcome when comparing to wired transmission technologies like Ethernet.

The first step to achieve a successful transmission is making sure the receiving radio is within transmit range of the transmitter. The
transmit range is decided by the power of which a signal is transmitted at, the antenna gain, and the surrounding 
environment. If there are a lot of solid obstacles, like walls and ceilings, the signal is likely to have a more compromised range.

Even if the surrounding environment is mostly open space, the wireless signal becomes subject of attenuation,
     which is a physical property of electromagnetic waves that weakens the signal the longer it has travelled through a
     medium. When this medium is only air, the phenomenon is referred to as free space path loss. Attentuation limits
     the transmit range of a radio, and to transmit further it has to increase its transmission power.

     \subsection{Collisions in wireless technology}
     Given two radio devices A and B, if there are any other nodes nearby that are within radio B's sensing range that sends at the same time as A, 
     radio B experiences radio frequency interference, and thus may not be able to correctly decode the signal of A. In 802.3 Ethernet this
     is graciously handled by collision detection in the CSMA/CD protocol. As each node can hear all other nodes on a wired medium,
     and listening while transmitting is generally not a problem, a node can retransmit if a collision is detected. In wireless techhnologies
     it is not equally easy to listen while transmitting, and collision detection may be impossible because of the hidden terminal problem.

     \subsubsection{The hidden terminal problem}
     The hidden terminal problem is one of the major challenges for wireless technologies, and a brief explanation of the problem will be offered here.
		 When node A transmits a message to node B, it may not be able to sense what is going on
     at the opposite side of node B. If a node C transmits at the same time, this signal may not enter node A's sensing range, and
     hence go undetected by, even though a collision has happened near node B. This is illustrated in figure \ref{fig:hiddenterminal} where
     both A and C has unknowingly sent messages at the same time, and B has not been able to decode either of the two, as they have been corrupted during
		 transmisison due to the collision. 

     \begin{figure}
     \center
		 \subfloat[A and C transmitting unknowingly at the same time, resulting in a collision at B]{{\includegraphics[width=13cm]{Images/HiddenTerminal.png} }}%

     \caption{The hidden terminal problem illustrated}
     \label{fig:hiddenterminal}
     \end{figure}


\section{Network infrastructure}
		This section provides a brief introduction to network infrastructures, to get an understanding of the differences between the mainly two types of infrastructures used today. 
		\subsection{Basic Service Set}
		A basic service set (BSS) in infrastruture mode consists of a single access point (AP) connected by wire to a distribution system (in residential deployment the distribution system is typically the ISP).
		Stations within the basic service area, which is the area physically serveable by the AP, can connect to the BSS using dynamic access point association, described
		in \cite{std80211}. A minimalist BSS infrastructure is illustrated in \ref{fig:basicserviceset}. It is also worth mentioning that a BSS can also be configured to operate as an independent
		basic service set, where there is no infrastructure in place (no APs), and communication is direct between stations. This is more commonly known as ad-hoc mode.
		
		One basic service set per household is the typical infrastructure in residential networks, but it is not uncommon to have multiple basic service sets in larger homes,
		where APs are placed on different floors to achieve better signal strength. Usually these APs have different service set identifiers (SSIDs), and each requires its own authentication.
		This is because the APs are not under the same extended service set. 

		 \begin{figure}
			 \center
			 \includegraphics[scale=0.35]{Images/BSS.png}
			 \caption{Minimalist Basic Service Set in infrastructure mode}
			 \label{fig:basicserviceset}
		 \end{figure}

		\subsection{Extended Service Set}
		An extended service set (ESS) can consist of multiple basic service sets, and thereby APs. The extended service set is a logical entity which can extend station
		authentication and association to all APs under the same extended service set. In an ESS the SSID of all APs are also identical. This means a station (STA) can be in the basic service area of
		multiple APs at the same time and is given the opportunity to select which AP to be serviced by. APs in an ESS can operate on different channels, and the medium access control
		mechanisms are performed as usual. Figure \ref{fig:extendedserviceset} illustrates the infrastructure of an extended service set.  

		 \begin{figure}
			 \center
			 \includegraphics[scale=1]{Images/ESS.png}
			 \caption{Extended Service Set}
			 \label{fig:extendedserviceset}
		 \end{figure}


     \section{MAC Layer}
		 In this section parts of MAC layer of the 802.11 protocol is accounted for.  
     The 802.11 MAC layer implements the Carrier Sense Multiple Access / Collision Avoidance (CSMA/CA) protocol to control access to the medium.
     The CSMA/CA protocol is designed to operate in an entirely distributed fashion, where all stations connected to the same basic service set operates on
     the same frequency without coordinated timeslots. As suggested by the name of the protocol, there are two basic operations in the CSMA/CA protocol:
     \textbf{Carrier Sensing (CS)} and \textbf{Collision Avoidance (CA)}, both of which will be briefly described here.

     \subsection{Carrier Sense}
     In 802.11, carrier sensing (CS) is done in two ways
     \begin{itemize}
     \item \textbf{Physical carrier sensing} handled by the physical layer (PHY) as Clear Channel Assessment (CCA), which we will talk about in the PHY subsection.
     \item \textbf{Virtual carrier sensing} is a MAC layer mechanism in place to limit the number of times
     a node has to request CCA from the physical layer. When a valid 802.11 frame is decoded on a listening wireless network interface, it can read the duration of
     the transmission from the MAC header. The frame containing a duration is called a Network Allocation Vector (NAV). When a NAV is received 
     the channel is marked as busy and the node will refrain from transmitting and also refrain from rechecking the channel for the duration of the NAV. 
     \end{itemize} 


     \subsection{Collision Avoidance}
CSMA/CA attempts to avoid collisions and is helpful in a network layout that includes hidden terminals. Request To Send/Clear To Send (RTS/CTS)
	is the function that allows CSMA/CA to some degree avoid the hidden terminal problem. By letting a node first ask the receiver if it is
	available for transmission (RTS), it prevents the node from sending the payload frame unless it receives Clear To Send (CTS) frame from the receiver first.
	The other mechanisms for collision avoidance are: 
	\begin{itemize}	
	\item \textbf{Interframe spacing} (IFS) is the amount of time the channel has to be idle before a sender can compete for channel access. 
	To give priority to certain frame types, different types of frames can have different types of interframe spacing. The type of IFS is usually 
	prefixed with the letter of the frame type. Organized by relative interval length, the differents IFS are:
	\begin{itemize} 
	\item Short IFS (SIFS), before ACK, RTS, CTS.  
	\item DCF Mode IFS (DIFS), before RTS frames (or DCF data frames if RTS/CTS is disabled)

	\end{itemize}
	\item \textbf{Exponential backoff} is what prevents two competing nodes from sending at the same time. When the channel is clear
	for DIFS time, a node has to wait another  random number of milliseconds before transmitting. This is called backoff.
	The amount of backoff is randomly chosen from a contention window (CW). The contention window has a low start size,
	called $CW_{\text{min}}$. A node draws a backoff time in the range $(0, 2^n*CW_{\text{min}})$, where $n$ is the number 
	of times the transmission have failed, beginning at $n=0$, and $CW_{\text{min}}<CW_{\text{max}}$.
	\end{itemize}

	\subsection{Distributed Coordination Function}
	The distributed coordination function is the main mode of operation in the 802.11 MAC layer, and is supposed to provide fair and reliable 
	transmission for all nodes on the same network. Figure \ref{fig:dcfmode} shows the 
	frame exchange that happens. Following the figure, the transmitter has to wait DIFS time, before drawing 
	a number from the contention window and backing off that amount.
	As no other transmissions has begun during this time, and the channel is
	still clear, the transmitter sends out an RTS frame. When received by the receiver
	it waits SIFS time before transmitting a CTS frame. The transmitter then sends his
	data frame, and waits for the ACK that indicates a successful transmission. The 
	RTS/CTS mechanisms introduces extra overhead and is sometimes turned off. The
	size of the payload and the number of stations on the network decides
	whether it is beneficial to have it on or off \cite{DCFanalysis}.



	\begin{figure}
	\center
	\includegraphics[scale=0.35]{Images/DCF.jpg}
	\caption{The timeline one frame transmission cycle in DCF mode}
	\label{fig:dcfmode}
	\end{figure}



	\section{PHY Layer}
	The physical layer in 802.11 is also divided in two sub-layers. The upper sub-layer is the Physical Layer Convergence Procedure (PLCP), responsible for clear channel assessment and acting
	as a common interface for MAC layer drivers. The lower sub-layer is the Physical Medium Dependent (PMD) which is responsible for modulation and directly interfaces with
	the radio. It is responsible for transmitting the complete frames, as well as receiving and decoding incoming frames. 

	\subsection{PLCP Protocol Data Unit}
	The physical layer convergence procedure creates PLCP Protocol Data Units (PPDUs), 
	that consists of 3 parts. The preamble, the header and the frame from the MAC layer
	called MAC Service Data Unit (MSDU). The frame structure
	has a long and short format, and changes a little bit for High Rate DSSS (HR-DSSS),
	but contains mostly the same fields. This is the structure of a PPDU header:

	%\begin{figure}
	%\center
	%\includegraphics[scale=0.5]{Images/PPDU.png}
	%\caption{DSSS PHY PPDU format from IEEE Std 802.11-2016  }
	%\label{fig:PPDU}
	%\end{figure}


	\begin{itemize}
	\item \textbf{Sync} bits are used to acquire the signal and synchronize timing. 
	\item \textbf{SFD} stands for Start Frame Delimeter and is there to indicate the start of a frame.  
	\item \textbf{Signal} the modulation type used to encode the MPDU and
	the data rate it is sent with. 
	\item \textbf{Service} Reserved for future use. 
	\item \textbf{Length} 16-bit fields that indicates the amount of time (in
			microseconds it will take to transmit the MPDU).
	\item \textbf{CRC} is the cyclic redundancy check that protects
	the fields signal, service and length. 
	\end{itemize}

	\subsection{Clear channel assessment}
	The PLCP layer provides clear channel assessment.
	The purpose of clear channel assessment is to give information to the MAC
	layer if the medium is available for transmission. There
	are primarily two ways the physical layer does CCA.
	\begin{itemize}
	\item CCA-ED (CCA-Energy Detect) detects signals that can not be decoded as a 802.11 frame, but is nonetheless a disturbing signal on the channel. If the CCA-ED value
	exceeds a threshold, for instance 20 dB, then the CCA shall be indicated as
	busy by issuing a \verb|PHY-CCA.indication(BUSY)| to the MAC layer.  
	\item CCA-CS (CCA Carrier/Sense) detects a valid 802.11 frame and can
	properly decode the header fields of a valid PPDU frame.
	The channel gets marked as busy for as long as the length
	field in the PPDU header specifies, even if the observed
	signal is weaker than the ED threshold. 
	\end{itemize}

	\section{Radio Frequency Interference}
	Radio Frequency Interference (RF-interference) is the result
	of two or more signals being transmitted on the same frequency at the same time.
	A receiver will have problems deciding which parts of the signals 
	belongs to which transmitter, and the signal may be altered to the extent
	that bits are changed or misrepresented. As the 2.4 GHz band that 802.11
	utilizes is a part of the Industrial, Scientific and Medical (ISM) band, channel noise or interference
	can come from sources such as microwaves, bluetooth devices and radio controllers, in addition to other Wi-Fi devices. 

	\subsection{Impact in 802.11}
	If the PPDU header
	gets corrupted by an interfering signal and can not be decoded
	, the PLCP layer issues a \verb|PHY-RXEND.indication(CarrierLost)|
	to the MAC layer. According to CSMA/CA the station then has to
	wait EIFS (Extended Interframe Spacing) time before
	it can transmit a new frame. EIFS is defined as
	\verb|ACK transmission time + SIFS + DIFS.| This is because the station that received the corrupt frame have no idea if any neighbouring station
	received it correctly, and is about to transmit an ACK-frame in response. Additionally
	to waiting the minimum EIFS time, the station also has to wait for
	an idle channel indication from the PLCP layer again. 

	This is just the added delay if a collision happens. Of course, the more co-channel interference, the more transmitters are waiting for medium access, meaning
	the contention window will increase, and the frequency of which a transmitter is granted access to the medium will be reduced. Thus, interference is severely damaging for QoS on wireless networks.   % The correct formula for EIFS is $DIFS + SIFS + lowest transmission time" 

	\subsection{Countermeasures}
	Several countermeasures to limit the impact of RF-interference have been suggested.
	On the MAC layer there is frame aggregation with individual headers.
	Frame aggregation in 802.11n is a technique that wraps several payloads under the same
	header and sends them together at the same time, once channel access is granted. This can improve the throughput
	if the channel is clear, but if the frame gets corrupted during transmission
	an increased amount of data is lost.
	Therefore it can be beneficial to aggregate a frame with individual headers.
	Even though this gives a slightly increased processing and transmission overhead
	compared to regular aggregation where there is no individual headers, 
	each frame can be selectively acknowledged.
	This means that only a few frames has to be retransmitted in case of corruption,
	and not the entire aggregation. 

	Additionally on the physical layer there are a couple of other suggested countermeasures:
	\begin{itemize}
	\item Changing the transmission power levels
	\item Lowering data rates
	\item Adjusting CCA threshold
	\item Forward error correction
	\item Changing packet sizes
	\end{itemize}


	\begin{wrapfigure}{R}{0.25\textwidth}
		\caption{Channel distribution}
		\label{fig:channels}
		\includegraphics[width=1\linewidth]{Images/Channels80211.png}
	\end{wrapfigure}


	The mentioned countermeasures only have limited
	impact, while switching to a channel with less interference (if available) remains the most effective in most cases \cite{impactRF}. 
	\section{Channels} 
	802.11 b/g/n uses the range of frequencies from 2.400-2.500 GHz on the ISM band.
	The increasingly popular 802.11n/ac also uses a range of 5 GHz frequencies on the Unlicensed
	National Information Infrastructure (UNII) band, which offers more frequencies \cite{5ghz}.
	Other than that the properties and challenges for the different bands are ultimately the same.
	The distribution of the frequencies on the 2.4 GHz band to the different channels is illustrated by figure \ref{fig:channels}. The frequencies listed are the center frequencies of each channel. In practice this means that an AP that transmits on one channel will interfere with close channels in both directions. Two channels are entirely non-interfering if they send on two frequencies  $f_{1}$ and $f_{2}$ so that $|f_{1} - f_{2}| > 0.025$. This means that there are in total three completely non-overlapping channels, 1, 6 and 11. The process of deciding which channel to transmit on is called channel allocation. 


\section{Relevant clustering methods}
This section introduces some of the clustering methods that will be used in this thesis.

\subsection {Hierarchical Agglomerative Clustering}
Hierarchical agglomerative clustering works in a bottom-up, greedy approach. This means that the algorithm
begins at the bottom node level, instead of considering the entire topology at once.
The starting number of clusters is the same as the number of points in the data set. Based on a specified \textit{distance metric} the two clusters that are closest to one another is merged into combined cluster. The clustering is complete
when there is only one cluster left \cite{murtycluster}. It falls under hierarchical clustering category, because the cluster subsets are iteratively being built for each merge and a dendrogram
that encompasses all sub-clusters can iteratively be constructed. An example of agglomerative clustering used on two dimensional data points can be seen in figure \ref{fig:agglomerative}
with its respective dendrogram. 

\subsubsection{Distance metric}
The distance metric lets us specify exactly what the distance between two clusters signifies. As each cluster usually consists of several nodes, there are multiple
options for what the distance between two clusters could be defined as. The distance metrics used in hierarchical clustering are typically either:

\begin{itemize}
	\itemsep0em 
	\item The average distance between the nodes of each cluster
	\item The minimum distance between the nodes of each cluster
	\item The maximum distance between the nodes of each cluster
\end{itemize}


\begin{figure}
		\centering
		\subfloat[Cluster formation]{{\includegraphics[width=6.5cm]{Images/agglomerative.png} }}%
		%\qquad
		%\qquad
		\subfloat[Dendrogram representation]{{\includegraphics[width=6.5cm]{Images/dendrogram.png} }}%
	\caption{Agglomerative clustering}
	\label{fig:agglomerative}
\end{figure}


%\subsubsection{Assessment}
%Agglomerative clustering provides us with the basic framework for an algorithm we can use to create access point groups. It begins by considering the nodes which
%observe each other as the closest nodes. However, there are two points where agglomerative clustering fails to meet our requirements: 

%\begin{itemize}
	%\item There is no upper limit to the amount of nodes in each cluster.
	%\item As the clustering algorithm will be running distributed, a decision has to be made with only the local distance observations at hand.
		%Assuming that there exists two groups that finds each other the closest (most disturbing), might prove a fallacy. Signal strengths are observed locally, and while one node observes another with a given signal strength, the mutual observation might be a different one. A situation like this could create a deadlock where there exists no combination of group pairs where both observes the other
		%as the closest group. 
%\end{itemize}

%The first point has the trivial solution of checking the number of nodes before accepting a merge,  and would only be a slight implementation change of agglomerative clustering. 
%The second point forces us to change the paradigm of always finding two mutually closest groups. 



\subsection{K-means}
The standard K-means algorithm, also called Lloyd's algorithm as it was first suggested by Seth Lloyd in \cite{sloyd}, is a clustering algorithm often used for cluster analysis, but also classification purposes in machine learning and unsupervised learning \cite{Coates2012}. K-means is an iterative algorithm designed to find a predefined amount of clusters in a data-set with unclassified data points. The amount of clusters to find is denoted by
K (hence K-means). 

In the default version of Lloyd's algorithm K centroids are randomly placed somewhere in the data set, then the distance between each data point and each centroid is computed. All data points
closest to the same centroid becomes part of the same cluster set. When all data points are associated with a cluster, each cluster computes the mean position of all its data points.
The value of the mean position becomes the location of the new centroids. When all the new centroids are computed, one iteration is over. The algorithm continues until the position 
of all centroids remains unchanged after an iteration. When the position of the centroids are the same two iterations in a row, it means that the solution has converged and all K-clusters has been identified \cite{mackaymeans}.

\begin{figure}[t!]
	\centering
		\subfloat[Original dataset]{{\includegraphics[width=4.5cm]{Images/kmeans0.png} }}%
		\qquad
		\qquad
		\subfloat[Two centroids (diamond shapes) randomly placed and cluster memberships calculated]{{\includegraphics[width=4.5cm]{Images/kmeans1.png} }}%


		\subfloat[The positions of centroids and the cluster memberships are updated]{{\includegraphics[width=4.5cm]{Images/kmeans2.png} }}%
		\qquad
		\qquad	
		\subfloat[The clusters have converged]{{\includegraphics[width=4.5cm]{Images/kmeans3.png} }}%
		\caption{Illustration of the K-means clustering algorithm}%
		\label{fig:mincutresults}%
\end{figure}


A well known problem with K-means is that the result strongly depends on how the algorithm is initiated. It always converges to the local minimum, and has no way of transcending this local minimum
This problem can be reduced by running K-means several times and choosing the best result of all the runs. 



%\chapter{Channel Allocation}

\begin{itemize}
\item Least Congested Search
\item MinMax
\item MinMax II
\item Hminmax/Hsum
\item Pick-Rand and Pick-First Approach
\item Pick-Rand and Pick-First II Approach
\item Channel Hopping Approach
\end{itemize}

\chapter{Related work}
This chapter is dedicated to give a brief account of works related to solving some of the challenges posed so far in this thesis. It is worth noting that some of the 
works mentioned here are actual implementations and live products, but only deployed in corporate environments.

\section{Cisco RRM}
CISCO offers a solution directed at enterprise networks \cite{ciscoRRM}, where implementing and managing a centralized
controller is a more managable task than in residential networks.
Their architecture consists of one or more Wireless Lan Controllers (WLCs).
The WLCs creates a group name and shares it with all APs in the same RF-group. All APs broadcasts their RF-group name, along
with the IP-address of their controller. Any other AP sharing the same RF-group name reports the incoming broadcast message to the controller, and then rebroadcasts the packet.  This way the controller
becomes aware of all APs that are in range of each other, similar to the flooding routing mechanism,
and they can form logical groups based on this information.
If applicable the RF-groups perform leader election to decide which controller becomes the RF Group leader, but there
can also a preconfigured leader. The RF group leader has the responsibility of running the relevant channel allocation
algorithms and ajusting the radio power level of the APs. 

\section{DenseAP}
DenseAP described in \cite{Murty2} aims to restructure the infrastructure of enterprise networks.  The two fundamental
changes they suggest is to deploy APs a lot denser (hence the name), and moving the task of associating clients
with APs to a centralized controller. The reasoning behind the dense deployment of APs is that signals diminishes
quickly in an indoor environment, and ideally a client should always be associated with an AP in the close vicinity. The argument 
they use for moving the association decision away from the client and over to a central controller, is that a client can
only use signal strength as the metric deciding which AP to associate with. This is emphasized as
suboptimal in conferences and meeting room environments, where many clients seek to associate with an AP at the same time.
If all clients pick the same AP it also means all clients will transmit on the same channel,
and RF-interference can reduce the throughput on the medium. 
Their infrastructure consists of DenseAP access points (DAPs) and DenseAP Controllers (DCs). The DAPs sends periodic
reports to the DC, which contains information as RSSI measures, channel interference, and associated clients. Based on this
information the DC decides which DAP each client should associate with, and also which channel each DAP should transmit
on. 

\section{HiveOS}
HiveOs \cite{Aerohive}, developed by Aerohive Networks offers distributed protocols and mechanisms to improve Wireless LANs in enterprise networks. The APs 
in the network are called HiveAPs, and they offer services such as
\begin{itemize}
	\item Band steering: if an device can operate on the 5Ghz band, it will be forced to connect to the 5Ghz network to optimize the utilization of the radio spectrum. 
	\item Load balancing: all HiveAPs have real-time information about how clients performs. If a client tries to associate with a new HiveAP, it will only be accepted
				if the new HiveAP has a low enough load to handle more clients. It would also know if other neighbouring APs are better suited to handle the load of the new client.
				This is achieved by witholding probe responses.

	\item Channel allocation: by using the Aerohive Channel Selection Protocol, HiveAPs tries to select the channel with the lowest co-channel interference. Their channel selection protocol uses 5 measurements, two static and 3 dynamic.
	The first static measurements is the number of nearby APs who are operating on the same channel. The more APs there are the higher the penalty. The penalty per AP diminishes as the number of APs increases, this
		is because the first one(s) are the most critical. The other static cost factor is what power level can be transmitted at the given channel, as this may differ on some 5GHz non-overlapping channels. The dynamic measurements are CRC-error rate, channel utlization,
	and the utilization of overlapping APs. All of these dynamic factors can penalize a channel with 0 to 3.5%. 
\end{itemize}

\section{ResFi}
ResFi \cite{resfi} is the only significant related work that directly aims to enable self-organized management in residential deployments of wireless LAN. Works such as \cite{Murty2}, \cite{ciscoRRM} and \cite{Aerohive} are all directed toward enterprise
networks. As we also aim to mainly concern ourselves with residential networks and their infrastrucure, ResFi is especially interesting to look at. ResFi assumes that all access points have two interfaces, one connected to a wired backone (e.g Internet), and another
802.11 compatible wireless interface. The figures of ResFi includes a Radio Resource Management Unit (RRMU). This is simply the device that interfaces with the antenna, and in most residential homes it will be a router that controls the channel and the power levels of the antenna. In short ResFi enables communication between access points under different basic service sets without imposing a central controller on the access points or being a part of an extended service set. More importantly, ResFi enables all of this without doing any modifications to hardware and drivers (like modifying standards or requiring propriatary equipment). 

\subsection{Operation}
This section contains a short introduction of the way ResFi is initiated and operates.
\begin{enumerate}
	\item An AP that has just booted up beacons a frame on all available channels. This frame contains: a globally routable IP-address, port, and two cryptographic keys. One transient key for group communication, and the public key for the RRMU. 
	\item All APs that can hear the beacon responds with a probe response containing the same information as in 1.
	\item When the scan is complete, the new AP can establish secure point-to-point communicate with all other APs using the wired backbone, the globally routable IP-address and the cryptographic keys
\end{enumerate}

\subsection{Implementation}
This section is dedicated to a brief overview of how ResFi can be implemented and deployed.

Originally ResFi was implemented on Ubuntu 14.04. It adds vendor specific information elements to the MAC-header with the IP, port, and cryptographic data. This can be done Linux user space by using their modified version of hostapd \cite{resfigit}.
As it runs on python, it could in practice be implemented on any Linux system. It uses a south-bound API to communicate with the RRMU, and a north-bound API to enable applications to use ResFi. ResFi itself provides no specific channel allocation mechanism nor a group-creation/AP-clustering algorithm, but the north-bound API could fascilitate applications that provides services like these.

%\section{Distributed Clusteering}
%DCA and DMAC

\section{Channel allocation using DSATUR and SCIFI} 
This section is dedicated to previous work done using the DSATUR heuristic. A little note to this section: it would also fit well in the background section. It accounts for the fundamentally important priniciple that
channel allocation can be treated as a graph coloring problem once a network graph is constructed.  

\subsection{DSATUR}
DSATUR (from degree of saturation) is a heuristic created by Daniel Brélaz \cite{Brelaz} to find solutions to the NP-complete problem of coloring the vertices of a graph so that no adjacent vertices share the same color. 
Channel allocation schemes relying on the DSATUR algorithm has been proposed before. In 2004 a paper was published, called
"Automatic channel allocation for small wireless local area networks using graph colouring algorithm approach" \cite{mahonen}, where the idea is to listen for neighboring AP beacon frames to create a list of all neighbours.
This list would be broadcast to all the neighbours. For multi-hop support, all receiving nodes will rebroadcast the list in a fashion equal to the flooding routing mechanism. This enables routers to create a graph of access points, and the DSATUR algorithm can then be used to compute the channel distribution. 

\subsection{SCIFI}
SCIFI \cite{SCIFI} is a centralized channel allocation protocol for infrastructure Wireless LANs that improves the traditional graph-coloring algorithm DSATUR. While it shows that their central coordinator
in fact improves the throughput compared to Wireless LANs that are not configured by SCIFI, the it is an algorithm for setting a channel in a preconfigured adminstrative domain. It does not deal with how the administrative domains
or router clusters are defined. If the method of creating clusters of collaborating routers proposed in this thesis has merit, SCIFI could plausibly be a supporting technology to compute channel distribution,
where a group acts as the administrative domain required by SCIFI. 



\chapter{Data acquisition and structure}\label{dataacc}
%
%Before implementing and testing any group creation algorithms, we need to gather usable data
%to perform testing on. The data consists of the topology size, and where in the topology a node is located.
%
%For basic testing we can create our own data, and either assume the location of nodes, or randomly distribute them on a topology.

We don't have the time nor the resources to organize a large enough testbed to purposefully evaluate the algorithms and suggestions that we will look at in the 
consecutive chapters. A sufficient testbed would require a large amount of routers (100-200 for a low scale test) in a small geographical area, preferably installed in 
residential apartment buildings. Not only would the creation of such a testbed require a lot of physical equipment, but there is also a lot of logical challenges 
that would have to be overcome, such as communication protocols and distributed consensus. These are problems we will address at a later point. Additionally, 
we could argue that it never advisable to go directly from an idea to a testbed anyway, especially without having any empirical indication of which approaches have merit,
and which is not worth the effort to implement. The risk is to waste a lot of time and resources to create a real life implementation that could have been identified as pointless 
in an early calculation or simulation. 

In this chapter we will consider how we can develop a tool that generates and visualizes the layout of network topologies based on constructed, artificial data, or
real-life location data of access points. 

\section{Requirements}
SSID, current channel frequency, radio power, physical data rate and supported 802.11 standards are just a few properties of a single real-life access point.
We are going to represent such access points in our network topology, but the access points we are representing will not be in the transmission (CSMA/CA) state where
most of theese properties are used. Our access points are in a state we can call the \textit{group discovery state}, where the goal is to find or create a group to be a part of
so that a channel can be assigned before transmission happens. Hence, to perform our computations, we don't have to consider many of these properties.
Actually, we will only store each access point's SSID, because this is a practical way to uniquely identify them \footnote{In the real world this is of course not the case,
but we will enforce it in our computations. We could just as well call it a unique id.},  and the the list of other access points that can be seen through a WLAN scan with
and their observed signal strength. These are the only metrics we require, but unfortunately there is no publicly available data source that contains the subjective radio scans
of a large amount of access points in the same area.

As a basis for our simulation we will represent nodes on a two-dimensional grid. Each dataset has to contain a set of nodes with two coordinates $x$ and $y$ so they can
positioned on the grid. When the nodes are placed on the grid they represent a network topology
and it will be possible to compute an estimation of which nodes can hear each other on the radio, and add these to each node's SSID-scan list.
This list contains the names of all the nodes it can hear, and how loud it is heard measured in $-dBi$.
Additionally the following parameters should be variable depending on each test scenario:

\begin{itemize}
	\item Topology size with the possibility to give variable width of x- and y-axis as input arguments. These unit of the axes is meters.
	\item Number of nodes to place on the topology
	\item Minimum distance between nodes (in meters). This is only to avoid unlikely placement and extreme interference of nodes that are placed on top of each other. 
	\item Minimum loudness measured in $-dBi$ for a node to account another node as a neighbour (e.g -100 is too low for anyone to hear).
\end{itemize}


	\section{Program design}\label{prog_design}
	The topology generation program consists of two main functionalitites.

	The first functionality is creating a topology and generate nodes which are uniformly
	and randomly positioned on the network topology. The size of the topology, the number of nodes and the minimum distance
	between nodes are properties that are passed as input arguments to the program.

	The second functionality is performing the calculation of which nodes can actually hear each other over the radio.
	We are assuming all nodes	are transmitting with equal strength, and that the environment is flat and obstacle free. 
	All the neighbouring nodes that can be heard by a node, is added to its list of neighbours, and it stores the $-dBi$ value so it later can be shared with
	the group. The interference levels between access points are calculated by iterating through every access point. For each node $N$ we record its x and y position,
	and then start a second iteration through the nodes. For each node $n$ in the second iteration we calculate the distance $d$ in
	meters between $N$ and $n$ using Euclidean distance. The formula for isotropic antennas is described by Friis \cite{Friis46}, and can be used to
	derive the formula for free space path loss \cite{FSPL} that is as follows:
\[
	FSPL(dB) = 10\log_{10} \left( \frac{ (4 \pi f d)}{c} ^2 \right) 
\]	
	Where $d = distance$, $f = frequency$ and $c=constant$. The constant $c$ is used to account for different units. We will use the meters to denominate the distance,
	and megahertz for the frequency. The resulting formula which will be implemented in the program is
\[
	FSPL(dB) = 20\log_{10}\left( f \right)  + 20\log_{10} \left(d\right) - 27.55
\]	
	As we have to compute the distance from all nodes to every other node, the topology generation program has $O(n^2)$ complexity. A simplified version of the code
	can be seen in figure \ref{fig:dbiCreation}. 

	The resulting program, written in Python 3\cite{Python3}, contains an importable \textit{topology class}. This way, for further testing we can use different data
	sources to get the positions of nodes, and only let the topology class compute the list of neighbours. 
	

	\begin{figure}[H]
		\begin{python}
#In topology class
def measureInterference(self):
 for nodeSubject in self._nodes:  
  for nodeObject in self._nodes:
    nodeSubject.calculateInterferenceTo(nodeObject) 

#In node class
def calculateInterferenceTo(self, nodeObject):
 if self == nodeObject:
  return
 dist = round(self.distanceTo(nodeObject))

#If  nodes have same coordinate, set high interference. 
 if (dist == 0):
  dBi = -40
 else:
  dBi  = self.measureDbi(dist) * -1

def measureDbi(self, dist):
 return (20 * math.log(self._frequency, 10)) + 
(20 * math.log(dist, 10)) - 27.55

			\end{python}
			\caption{Computing the interference between nodes}
			\label{fig:dbiCreation}
			\end{figure}

			\section{Data output and visual representation} \label{simulationrep}
			The result of the topology generation is stored in a JSON\cite{JSON} data file. The data file contains the height and width of the the generated topology, as well
			as the number of nodes. A \verb|nodes| object consists of as many JSON node-objects as there are nodes. Figure \ref{fig:nodeStruct} illustrates the node structure
			and is an example of how a node with two neighbours will look.
			\begin{figure}[H]
			\begin{minipage}{\linewidth}
			\begin{lstlisting}[language=json]
{
  "mapWidth": 100,
  "mapHeight": 100,
  "nodeCount": 3,
  "nodes": {
    1: {
    "posX": 50,
    "posY": 50, 
    "ssid": "NODE1", 
    "neighbourCount": 2, 
    "neighbours": {
      0: {
        "ssid": NODE2,
        "dbi": -77.23
        },
      1: {
        "ssid": NODE3,
        "dbi": -79.52
        }
      }
    }
  },
...
}
\end{lstlisting}
\end{minipage}
\caption{JSON output structure}
\label{fig:nodeStruct}

\end{figure}
The program can be run in the following way
\begin{verbatim}./GenerateTopology.py -n 500 -w 100 -h 100 --space 10 --dbi 85 \end{verbatim}
Which instructs the program to create a topology with 500 nodes. The topology should be 100 by 100 meters large, and there should be at least 10 meters
between each node. The $dbi$ parameter makes sure that only nodes which can be heard with a $-dBi$-value of $-85$ or larger should appear in the neighbour list. 
The output from this specific run is a 23.2 MB large file containing the resulting topology-data in JSON.

By writing a simple HTML and JavaScript browser application, we can parse the JSON and visually represent the nodes on a grid.
The result will look like what can be seen in figure \ref{fig:randtop}

\begin{figure}[h]
\center
\includegraphics[scale=0.35]{Images/interface.png}
\caption{Generated topology with random, uniform distribution and}
\label{fig:randtop}
\end{figure}

\section{WiGLE}
WiGLE (Wireless Geographical Logging Engine) \cite{wigle} is a project started in 2001 which purpose is to gather information about wireless networks. All the information
they collect is user submitted stored in their database. Anyone can download an Android app developed and published by WiGLE, and use the app for wardriving,
\footnote{Wardriving is the act of tracking wireless networks using a laptop or a phone,	and then store the information about each network.},
	then submit the data to WiGLE's centralized database. The APs discovered can be viewed on an interactive map provided on WiGLE's website. 
	All the data can also be accessed through a public API which is especially interesting for us. Using their service is entirely free, but the
	amount of data that can be requested is throttled on a day-to-day basis. In the FAQ section on the website its written that the project openly supports research projects, 
	so after contacting them they upgraded the account we will be using for the data requests to a higher daily data quota.

	As the physical location of each access point is estimated using weighted-centroid trilateration, 
	we can fetch the estimated position of each access point. However, the location data is not necessarily always accurate: if the measurements of an access point signal
	strength is done in multiple places (in a way that a line between the measurement locations surrounds the access point), the access point location becomes very precise. 
	On the other hand, if the measurements are taken on only one side of the access point, the measurements are heavily shifted towards that side. {{figure?}} 
	
	Even though the data is not perfect, it is more suited to impersonate the layout of real-life network topologies and give us a better understanding of how what access point
	clustering would look like in real-life. 



	\subsection{REST API}
	WiGLE has a REST API that can be used to retrieve information from their database. The API responds to requests with JSON data. It gives access to a
	number of services such as user profile management, large-scale statistical information about the collected data
	and more specific network searches. An interactive guide to the public API is available at \cite{WigleAPI}. 

	We are going to mine data for the topology generation program, hence the network search service is all we are interested in. There are several ways an access point can be retrieved using
	the network search. We can query specific SSIDs, a range of coordinates, only APs with a minimum signal level, or only query networks that are free to use. As 
	we are only rebuilding the network topology to fit our tailored data structure, we will consider all access points and not filter on any parameters except for coordinates.
	By issuing an API a request for nodes between two latitudes and two longitudes, the response is an array with AP objects within that area.  
	A request will look something like what can be seen in figure \ref{fig:wigReq}.
	\begin{figure}
	\scriptsize
	\begin{lstlisting}[breaklines]
	 https://api.wigle.net/api/v2/network/search?first=0&latrange1=37.80846&latrange2=37.7467&longrange1=-122.5392&longrange2=-122.3813&start=0
	\end{lstlisting}
	\caption{Example of a Wigle API request}
	\label{fig:wigReq}
	\end{figure}

	The parameters $latrange1$ and $longrange1$ are the coordinates that marks the beginning of the area we are interested in and $latrange2$ and $longrange2$ marks the end. 
	We want information about the position of all access points within this range, however WiGLE returns at most 100 results per query. If there are more than 100 access points, we need to change the $start$
	parameter. This parameter tells WiGLE at which offset we want to begin fetching data from. For instance a start value of $0$ means we fetch the first 100 access points, with indexes 0-99. A value of $100$ means that
	we fetch the next 100 APs in the range $100-199$ and so on. The JSON response for a succesful request for one AP can be seen in figure \ref{fig:wigle}. 

	\begin{figure}[h]

	\begin{lstlisting}[language=json]
{
	"userfound": false,
	"qos": 0,
	"comment": null,
	"lastupdt": "2015-12-22T17:49:34.000Z",
	"bcninterval": 0,
	"dhcp": "?",
	"lasttime": "2015-12-22T17:49:15.000Z",
	"trilong": 10.82792618,
	"netid": "5C:9E:FF:2B:54:84",
	"freenet": "?",
	"trilat": 62.2816925,
	"name": null,
	"firsttime": "2015-12-22T20:55:01.000Z",
	"type": "infra",
	"ssid": "NETGEAR23",
	"paynet": "?",
	"wep": "2",
	"transid": "20151222-00207",
	"channel": 52
}

\end{lstlisting}
\caption{REST API response with AP data}
\label{fig:wigle}
\end{figure}

The JSON-object in the response contains quite a lot of information about all of the APs retrieved. We wont be using most of this data, but it is worth noting that some of it could be used to perfect the search. For instance the last updated parameter could
be a way to refine the access points retrieval, so that access points that has been long gone is not taken into consideration. The most valueable properties for us is $trilong$ and $trilat$. As the names suggest these contain the position calculated
by the weighted-centroid triliteration. 

\subsection{The Haversine Formula}
As seen in the previous section, WiGLE provides data about the location of access points and a way for us to retrieve this data in a usable format. 
At this point we could create a program that operates directly on the longitudes and latitudes retrieved.
This program would also need to be able to use the FPSL formula on global coordinates to compute the radio scans of each AP,
and to properly visualize the data we would have to reinstall the nodes on a globe. This implementation requires a lot of extra labour, and seems especially
redundant when we already have a working program that already has the aforementioned functionality.
The problem is that our previous work relies on a two dimensional Cartesian coordinate system.
To be able to reuse what we have have already built, the geographic coordinates has to be translated into coordinates in Euclidean space.

The haversine formula \cite{virtues} is used to accurately compute the great-circle distance between two global coordinates.

\[
	d =2R sin^{-1} \left(\sqrt{ \left(sin\left(\frac{\varphi_2-\varphi_1}{2} \right)\right)^2 + cos(\varphi_1) * cos(\varphi_2) * sin\left(\left( \frac{\lambda_2 - \lambda_1}{2} \right)\right)^2} \right)
	%a = sin²(Δφ/2) + cos φ1 ⋅ cos φ2 ⋅ sin²(Δλ/2)
	%FSPL(dB) = 20\log_{10}\left( f \right)  + 20\log_{10} \left(d\right) - 27.55
\]	

Where $d$ is the distance between two latitudes $\varphi_1$ and $\varphi_2$ and two longitudes $\lambda_1$ and $\lambda_2$. $R$ is the radius of the
sphere, which in this context is the earth's radius. 

This can also be expressed with a two-parameter inverse tangent function \cite{chamberlain_2017}, as long as neither of the
parameters are zero. 

\begin{flalign}
	\nonumber a &= \left(sin\left(\frac{\varphi_2-\varphi_1}{2} \right)\right)^2 + cos(\varphi_1) * cos(\varphi_2) * sin\left(\left( \frac{\lambda_2 - \lambda_1}{2} \right)\right)^2 \\
	\nonumber c &= 2*atan2(\sqrt{a}, \sqrt{(1-a)}) \\
	\nonumber d &= c * R
	%a = sin²(Δφ/2) + cos φ1 ⋅ cos φ2 ⋅ sin²(Δλ/2
	%FSPL(dB) = 20\log_{10}\left( f \right)  + 20\log_{10} \left(d\right) - 27.55
\end{flalign}


The python implementation of the formula can be seen in figure \ref{fig:haversine}. It is imortant to note that the degrees taken as input
has to been converted to radians before inserting it in the formula. 

	\begin{figure}[H]
		\tiny
		\begin{python}
def distanceBetweenCoordinates(lat1, lat2, long1, long2):
	deltaLon = math.radians(long2 - long1)
	deltaLat = math.radians(lat2 - lat1)
	a = (math.sin(deltaLat / 2))**2 + math.cos(math.radians(lat1)) * math.cos(math.radians(lat2)) * (math.sin(deltaLon/2))**2
	c = 2 * math.atan2(math.sqrt(a), math.sqrt(1 - a)) 
	R = 6371000
	d = round(c * R)
	return d
	\end{python}
			\caption{Implementation of haversine distance}
			\label{fig:haversine}
	\end{figure}


This can be used to get the distance in meters between the boundaries of our area of interest. When the data is retrieved from
WiGLE we can use the specified coordinates to compute the size of the cartesian coordinate system.
The distance between the first latitude $\varphi_{start}$ and the second latitude $\varphi_{stop}$
is calculated. This is done by using the same longitude  $\lambda_1 = \lambda_2$, and returns the size of the x-axis in meters.
The opposite has to be done to get the size of the y-axis, using $\lambda_{start}$ and $\lambda_{stop}$ and equal longitudes.
The same approach can be used to place each node on the coordinate system, where the first set of the coordinates is
$\varphi_{start}$ and $\lambda_{start}$, and the second set is the coordinates of the AP.
\subsection{Data output}
The resulting python program takes an output filename and two latitudes and two longitudes as input. It queries WiGLE's API for all the APs between these coordinates.
As long as the number of APs in the coordinate range does not exceed the daily data limit, all the data will be stored in a topology datastructure imported 
from the topology generation program in section \ref{prog_design}. The APs will be placed correctly relative to each other, with real world distance between them,
based on the coordinates of each AP. 

{{wigle data figure, overlaid map}}


\chapter{Access point clustering} \label{chap:clustering}
%To enable collaborative channel allocation, it is
%important for every AP to know which APs it is collaborating with.
%One way to share information about who collaborates with who,
%is to let the access points group together. Information relevant for channel
%allocation can then be shared freely within the group, between APs.

%We will proceed by looking at some of the requirements for a group creation algorithm.
%It should work decentralized in a distributed fashion. Hence, not only does the APs have to
%be imposed group membership, but they also have to be able to create and definine
%meaningful groups on their own. Later we will propose an algorithm to create groups,
%and then evaluate computed groups based on the algorithm.


\section{Introduction}
While it at first sounds incredibly desirable to let the entire population of for instance New York's access points organize themselves in an optimal channel-plan,
at second thought the idea may prove to be a little ambitious. Being limited by the NP-complete nature of DSATUR or similar graph coloring algorithms
we have to set some reasonable constraints on the size of the \textit{collaborating group}. The amount of nodes that will collaborate on the problem of finding an optimal channel distribution
plan may not need to be very high.

Let us for a moment look at an ideal example: a city that only consists of small apartment buildings that are entirely isolated from interference produced in external networks.
How could we build a group in such a case? Turns out it is not that hard. Each access point would be able to collaborate with every other observable access point.
When computing channel distribution there would be no risk of surpassing the viable amount of nodes to compute the distribution for, as the group is limited to the apartment building. Sadly for us the world is not ideal,
and it is not impossible that every access point in New York can observe eachother through a transitive relation. This means the size of
the collaborating group would be vast and completely unviable.  

This does not mean that creating reasonably sized groups of access points is impossible, but it poses a more difficult challenge. 
We need to filter out redundant nodes to create an approximated version of the ideal example.

Having a picture of the typical cityscape in mind, we know that buildings are naturally
separated by streets, bridges, parks, and so on. Rural areas are similar, but networks are spaced more unevenly
and usually only affecting each other in one plane.

Remembering attenuation, we know that RF-interference is also a property of distance. 
These pieces of information tells us that signal strength measured between access points in two separated buildings should be lower than readings between access points in adjacent apartments.
In the following chapter we will consider how we can utilize the signal strength measured between access points to build a clustering algorithm that creates reasonable groups of access points. 

%The should consist of APs that interfere with each other when their channels are overlapping, so that overlap can be avoided with a channel allocation algorithm run within the group. Not all APs in
%the connected group will necessarily be able to hear each other on the radio directly, but all nodes
%should be able to hear each other through a transitive relation (neighbour of a neighbour, etc.).

\section{Problem overview}
The main problem we are dealing with is finding a way for access points to group together in clusters that are geographically close to each other. This can either be solved with a centralized
coordinator or with a peer-to-peer distributed protocol. In this thesis we will be focusing on the distributed approach, but first we will take a look at the pros and cons of each approach.

\subsection{Centralized model}
Let us briefly envision how a centralized model might look. As we want to propose a solution that works across private consumer networks as well as larger corporate networks,
we can not assume that all networks are under the same administrative domain. This is where the centralized approach is limited, and why most solutions presented in the related work section is usually only deployed under the same extended service set (ESS).

Just like a distributed version, a centralized controller is also restricted by the NP-complete nature of all graph colouring algorithms.
To be able to compute a channel plan using heuristics, it has to identify groups of nodes that impacts each other severely. 

The advantage of the centralized model is the ability to have the full picture of access points readily available. Let us assume the controller is placed at an ISP or a router manifacturer.
The controller could potentially know exactly where in the world the access points are placed. Creating groups would be as simple as dividing within naturally separative geographical barriers like roads, streets and buildings. 

For a central controller to be able to identify APs that are near eachother and compute the optimal channel distribution, APs need to report their radio readings to the controller. 
The controller then needs to identify which of all the observed APs it can control, and which is not controllable and has to be treated as noise. There would be no requirement for
communication in between nodes, which reduces complexity a lot.  

A major drawback of the centralized approach is that nodes that are near eachother have to be under the same controller. If there are too many nodes around
that can not be controlled by the controller, all nearby access points would be treated as noise and regular channel allocation schemes would have to be applied.
Even though in there is a tendency for apartments in the same building to have the same ISP in Norway, there is no gurantee for that to always be the case.  

\subsection{Distributed approach}
We will continue considering a distributed approach as the main topic of the thesis. In reality this idea has some major challenges that we have to overcome. Here are the most prominent ones: 
\begin{enumerate}
	\item The group creation itself. Based on the signal strength and the communication channel, nodes have to be able to organize themselves in a tight cluster of nodes that
		includes all nodes that impacts each other most severely. This problem essentially boils down to a clustering problem. This is the main issue we will
		try to find a reasonable solution for. 
	\item The communication channel. Nodes have to be able to communicate with eachother. The 802.11 standard describes no protocol for communication between access points that are not on the same extended service set. This communication would have to happen on the network layer, preferably over TCP. This communication channel would have to convey a messages that enables group creation,
		most likely a custom protocol. 	
	\item State synchronization. As all nodes have to compute a channel plan for the group, they all need to have synchronized information about the members of the groups
		and possibly also every other node's signal strength measurements. 
	\item Channel plan computation. Even though the centralized approach would have a similar problem, it will be more difficult to solve in a distributed fashion as 
		it is reasonable to assume that an access point has limited computational power. This point extends past the scope of the thesis, but algorithms like DSATUR {{(insert ref)}}
		and SCIFI are examples of algorithms that computes channel plans as a graph colouring problems.
\end{enumerate}

\section{Computation assumptions and requirements}
Moving forward in the next chapters we will be looking at possible ways for nodes to organize themselves into groups.
We will be using the data we fetched in chapter \ref{dataacc}. The data provides a topology of nodes where all nodes have a list of neighbouring nodes.
This neighbour list contains the appropriate computed signal strength measurements for each node. Based on the signal strength of their neighbours, the nodes should be able to create groups that resembles
clusters. It is important that the border between two different groups are placed in such a way that interference between the two groups are as minimal as possible. To simplify the computation procedure and to reduce the workload for building the simulation program we are going to do a number of assumptions. 

    \begin{enumerate}
    \item Assumption 1: All nodes involved in the simulation also run the group creation algorithm. 
		\item Assumption 2: When a node is observed over radio, it is also known how to directly contact the node (e.g. via TCP). This assumption also implies that there is implemented a protocol that lets nodes communicate and exchange information about their signal strength measurements and group membership. For now we will assume this is true, but in chapter {{(ref)}} we will address the issue. 
		\item Assumption 3: All nodes in a group are completely synchronous, and always have an equal image of the state of the group at a given time. Sharing information with the group is instant. This is also an issue that will be addressed in chapter {{3}}.
    \end{enumerate}
\section{Program design}
\subsection{Design choices}
The program is designed to be modified so it can accomodate different algorithm types without changing the fundamental framework.
Everything regarding group computation is implemented in Python 3 \cite{Python3}, and is designed to parse the output from the data generation program and use it directly.
Group computation should be a fast operation, while the data generation (or fetching) is a slow process and should only be done once every time a new data set is required.
The program is not parallelized to keep results consistent and to easier debug and locate program errors. 

\subsection{Group framework}
The group framework consists of 3 classes with different responsibilities:
\begin{itemize}
	\item \textbf{Group}. An object of this class is an abstraction of all the nodes that belong to the same group. Because we assumed that all nodes have equal
	information about group membership and signal strength measurements, we can store all the members of each group in a list and let the group object act as a unified entity on
	behalf of the entire group. All the interfaces and logic for forming groups is placed within this class. A method named \verb|iteration|
	has the responsibility of triggering the appropriate action based on the state of the group. For instance adding nodes, removing nodes or merging the group
	with another. This is the part of the code that has to be changed when implementing and changing algorithms. If an action was performed the method returns 1, else it returns 0. 

	\item \textbf{GroupCollection}. An object of this class contains all the groups used in a simulation. Its main functional responsibility is looping through all the groups
	and calling the \verb|iteration| method of each group once. This is done in the GroupCollection's \verb|iterate| method. It accumulates the amount of changes done in all the groups,
	by adding the return values of the Group object's \verb|iteration| method. It also handles the destruction of groups, and bootstrapping of newly created groups.

	\item \textbf{Simulation}. An object of this class handles the bootstrapping of groups, where all nodes (given in the input file) are parsed and
	put in their own grown group. Consequently, at the beginning of each computation the amount of groups is equal to the amount of nodes.
	The Simulation class is also responsible for starting and stopping the simulation itself. After bootstrapping all the groups, the Simulation enters a loop
	where it calls the \verb|iterate| method in the GroupCollections object once every run. All the groups have converged and reached a steady state 
	once the amount of changes returned by the GroupCollection's \verb|iterate| method is equal to zero. The results are written to file.
\end{itemize}
{{flowchart}}
\subsection{Output file structure}
The results of the group computations are writen to file. The results does not only contain the resulting divison of groups, but to be able to recreate the simulation
visually, the results contains the topology of all nodes and their group membership for each iteration in the simulation process. This means that we can step-by-step
recreate the simulation. The data is stored as JSON, and the structure can been seen in figure \ref{fig:jsongroup}.
Having the data stored as JSON means that the data is language independent. This allows us to either implement a parser in python for the data and use \verb|matplotlib| to visualize it,
or we can use another applicaton to visualize the data. Since we already have the topology visualizer
written in HTML and JavaScript from chapter \ref{dataacc}, we can extend this program to let us upload a group creation output file. 
\begin{figure}

\begin{minipage}{\linewidth}
\begin{lstlisting}[language=json]
  "iterations": {
    "0": {
      "0": {
	"groupName": "GROUP0",
        "members": {
          "0": "NODE0"
	},
	"memberCount": 1
    },
      "1": {
	"groupName": "GROUP1",
	"members": {
	  "0": "NODE1"
	},
	"memberCount": 1
      }
  },
    "1": {
      "0": {
	  "groupName": "GROUP1",
	  "members": {
	    "0": "NODE0",
	    "1": "NODE1"
	},
	"memberCount": 2
     }
  }
}
\end{lstlisting}
\end{minipage}

\caption{Group simulation file structure}
\medskip
\small
This particular simulation had two iterations. In the first iteration there were two groups, each with one member node. In the second iteration the two groups has merged to one group, now
containing both nodes. 
\label{fig:jsongroup}
\end{figure}



%\section{Algorithms}\label{algorithm}
%We will consider three possible approaches to cluster access points. First we will look at a minimal and basic approach without any complicated heuristics or algorithms. 
%Next we will consider a well known clustering algorithm and treat the problem as a pure clustering problem. Finally we will look at the problem as a graph partitioning problem.
%All approaches will be tested with the same topologies: one uniformly distributed, one used for evaluating traditional clustering algorithms, and two topologies 
%of towns fetched from Wigle. Henceforth we will for the sake of simplicity be referring to APs that are running the group algorithm as \textit{nodes}.

\section{Creating groups with clustering}
In this section we will consider and evaluate a way to divide access points in groups by assessing an exisiting clustering algorithm, and consider different modifications to this algorithm
to make it better suited for a distributed clustering of access points. 

\subsection {Agglomerative Clustering}
There are a couple of reasons for choosing hierarchical agglomerative clustering as a starting point. First of all it does not need a pre-specified amount of clusters to be able to run,
like for instance partitioning clustering needs. The second reason is that agglomerative clustering has a bottom-up approach, where the clustering begins by treating nodes as clusters and iteratively
merges clusters to create a dendrogram that represents the current cluster and all previous merges. The bottom-up approach intuitively seems to fit well in a ditributed model, where all nodes
know only of themselves and their neighbours. 
\subsubsection{Description}
Agglomerative clustering  \cite{agglomerative} is a variant of hierarchical clustering where clusters are pairwise greedily merged. The starting number of clusters is the same as the number of
points in the data set. Based on a specified distance metric, the two clusters that are closest to one another is merged into combined cluster. The clustering is complete
when there is only one cluster left. It falls under the category of hierarchical clustering, and a dendrogram is iteratively being built for each merge,
containing all past merges of the clusters. An example of agglomorative clustering can be seen in {{figure}}\ref{fig:agglomerative}.

\subsubsection{Assessment}
Agglomerative clustering provides us with the basic idea for a an algorithm we can use to create access point groups. However, there are two points where agglomerative clustering
fails to meet our requirements: 
\begin{itemize}
	\item There is no upper limit to the amount of nodes in each cluster
	\item As the clustering algorithm will be running distributed  on each cluster, a decision has to be made with only the local distance observations at hand.
		Assuming that there exists two groups that finds each other the closest (most disturbing), might prove a fallacy. Signal strengths are observed locally, and while one node observes another with a given signal strength,
		the mututal observation might be a different one. This could create a deadlock where there exists no combination of group pairs where both observes the other
		as the closest group. 
\end{itemize}

The first point has the trivial solution of checking the number of nodes before accepting a merge,  and would only be a slight implemention change of agglomorative clustering. 
The second point forces us to change the paradigm of always finding two mutually closest groups. 

\subsection{K-Nearest Neighbour Clustering}
\subsubsection{Description}
To resolve the problems discussed in the previous section, we will create a slighly different algorithm and refer to it as K-Closest Neighbour Clustering (KCNC). 
In this method, similar to agglomerative clustering, in the beggining there are equally many clusters as there are nodes. Instead of looking for pairs that are mutually close,
each cluster seeks to merge with the cluster that is closest. The distance is defined by the distance metric. The merge will always happen as long as the resulting cluster does 
not contain a higher node count than K.

\subsubsection{Distance metric}
The distance metric lets us specify exactly what the distance between two clusters signifies. As each cluster usually consists of several nodes, there are multiple
options for what the distance between two clusters could be defined as. The distance metrics used in hierarchical clustering are typically either:

\begin{itemize}
	\itemsep0em 
	\item The average distance between the nodes of each cluster
	\item The minimum distance between the nodes of each cluster
	\item The maximum distance between the nodes of each cluster
\end{itemize}

The metric we will use is the -dBi value of neighbouring nodes. Every node in a cluster tracks the observed -dBi value of all other nodes that are not a member of the cluster.
The cluster containing the node with the highest -dBi value will be merged together with original one. This is similar to the minimum distance metric. 

\subsubsection{Implementation}
Each node begins by identifying itself as a member of a group that only contains itself.  Let us call this group $a$.
Group $a$ loops through the radio readings of every member of the group, and picks the node with the highest observed -dBi value
to contact. In the beginning there is only one node in group $a$. Hence in the first iteration this node's radio readings alone will decide which group to merge with. 
The neighbour node, which we will call $B$, is the node that disturbs group $a$ the most. $B$ is a member of group $b$.
In other words, group $a$ wants to merge with group $b$ to create a larger group that contains node $B$.

A merge happens in the following way: the members of the two groups exchange information about all their member nodes and their radio readings and combine the information.
As the data is now identical for all the members of both groups and they can make identical choices it means that they are part of the same group. 

In our simulation this is as easy as combining two Group objects into one. A pseudocode sample of an implementation can be seen in \ref{fig:groupmerge}.
In a real world implementation there would have to be a supporting protocol to 
enable the flow of information and synchronization of data. 

	\begin{figure}
		\tiny
		\begin{python}
allGroups = [];
K = 120;

for node in topology: //Initialize groups
	g = new Group()
	g.members.append(node)
	allGroups.append(g);

while True: //Run as long as there are changes
	changes = 0;	
	for group in allGroups: 
		groupMaxDbi = -INFINITE
		groupClosestNode = None

		for node in group:
			nodeMaxDbi = -INFINITE
			nodeClosestNode = None

			for neighbour in node.neighbours:
				if neighbour.dbi > nodeMaxDbi:
					nodeMaxDbi = neighbour.dbi
					nodeClosestNode = neigbour

			if (nodeMaxdbi > groupMaxDbi):
				groupMaxDbi = groupMaxDbi
				groupClosestNOde = nodeClosestNode:

		var groupToMergeWith = groupClosestNode.group
		if (length(group.members) + length(groupToMergeWith.members)) < K:
			var newGroup = new Group()
			newgroup.members = group.members + groupToMergeWith.members
			allGroups.append(newgroup)
			allGroups.remove(groupToMergeWith)
			allGroups.remove(group)
			changes = 1
	if (changes == 0):
		break
		\end{python}
			\caption{Pseudocode sample of how the K-Cloest Neighbour Clustering runs in a simulated environment}
			\label{fig:groupmerge}
	\end{figure}



We can not always accept merges, else we would end up with a group that spanned the entire topology. That is why we define a maximum threshold for the amount of members a group can have, 
referred to as K. If the sum of members in two groups that wants to merge exceeds K, the merge is aborted and no changes is reported to have happened for either group. 
This means that the simulation algorithm converges when no groups remain that are small enough to merge with another.
 
\subsection{Results on different topologies}
Figure \ref{fig:knearest}a shows the result of running the K-Closest Neighbour Clustering algorithm on a uniform distribution topology, while figures \ref{fig:knearest}b, \ref{fig:knearest}c, \ref{fig:knearest}d shows the resulting clusters after running it on topologies on citites and towns collected by WiGLE. The different node colors indicate different group memberships. The large scale images of each topology can be seen in appendix \ref{appendix:knn}.


\begin{figure}
		\centering
		\subfloat[Uniform]{{\includegraphics[width=4.5cm]{Images/computations/BASIC500x500_1000n.jpg} }}%
		\qquad
		\qquad
		\subfloat[Forks]{{\includegraphics[width=4.5cm]{Images/computations/BASICForks.jpg} }}

		\subfloat[Lillehammer]{{\includegraphics[width=4.5cm]{Images/computations/BASICLillehammer.jpg} }}%
		\qquad
		\qquad
		\subfloat[Tynset]{{\includegraphics[width=4.5cm]{Images/computations/BASICTynset.jpg} }}%
		\caption{K-Closest Neighbour Clustering on different topologies}%
		\label{fig:knearest}%
\end{figure}



%\begin{figure}    
%	\centering
%	\begin{minipage}[t]{0.3\textwidth}
%		\scriptsize
%		\includegraphics[width=\linewidth]{Images/computations/BASIC500x500_1000n.jpg}
%		\caption{KNNC: Uniform Distribution}
%		\label{fig:immediate}
%	\end{minipage}
%	\hspace{0.5cm}
%	\begin{minipage}[t]{0.3\textwidth}
%		\includegraphics[width=\linewidth]{Images/computations/BASICForks.jpg}
%		\caption{KNNC: Forks, USA}
%		\label{fig:proximal}
%	\end{minipage}
%
%	\vspace*{0.5cm} % (or whatever vertical separation you prefer)
%	\begin{minipage}[t]{0.3\textwidth}
%		\includegraphics[width=\linewidth]{Images/computations/BASICLillehammer.jpg}
%		\caption{KNNC: Lillehammer, Norway}
%		\label{fig:distal}
%	\end{minipage}
%	\hspace{0.5cm}
%	\begin{minipage}[t]{0.3\textwidth}
%		\includegraphics[width=\linewidth]{Images/computations/BASICTynset.jpg}
%		\caption{KNNC: Tynset, Norway}
%		\label{fig:combined}
%	\end{minipage}
%\end{figure}
%
\subsection{Evaluation}
By looking at the results of the simulations of the K-Closest Neighbour Clustering it is obvious that clusters are created.  Even in the more chaotic topologies as Lillehammer,
there are distinct clusters that encompasses the closest nodes. However, there are two obvious problems with this algorithm:
\begin{itemize}
	\item In contrast to agglomerative clustering, our K-Closest Neighbour Clustering algorithm does not care if the closest distance between the groups is pairwise mutual. This
		might lead to one group $a$ merging into another group $b$, even while $b$ has another neighbouring group $c$ that lies closer. If the maximum size K is reached during the merge of
		$a$ and $b$,  the resulting group will never be able to merge with $c$ even though that would have given a better cluster. 
	\item Once a cluster reaches the maximum size K, there is no possibilty for any other nodes to join the cluster.
		In our simulation of static topologies that might work, but in a real world scenario access points may turn on randomly, and be located in the middle of an already existing cluster.
		With the current algorithm there is no way to handle this event, and the node could not become a member of the cluster that surrounds it. 
\end{itemize}

In the next section we will look at how we can use group splitting to create a solution for these problems. 


\section{K-means splitting}
In this section the the concept of group splitting will be introduced. Group splitting is thought to improve the K-Closest Neighbour Clustering algorithm suggested in the previous section.
Moreover, this section is dedicated to the introduction, simulation and evaluation of one of the two split approaches that will be taken in the thesis.

\subsection{Group splitting}
An implication created by letting one cluster merge with another without knowing if the merge is mutually beneficial, is that not all merges will be optimal.
A way to increase the likelihood of a group ending up with the neighbouring nodes that impacts each other severely would be to accept merges that result in groups larger than K, the maximum size. 
The group would be too large, but more likely to have the opportunity to merge with the nodes of highest impact. To reduce the group size back to K,
the least impactful nodes would have to be kicked out of the group. This is what we call group splitting. A simplified illustration of the basic concept is shown in figure \ref{fig:splitting},
where the maximum group size is 2 and two neighbouring nodes join to reach the maximum group size. A third, more impactful node appears, and is included in a transient group
before the least disturbing node is kicked out to get the group size back to K. 

\begin{figure}
	\centering
		\subfloat[Two neighbouring nodes]{{\includegraphics[width=5cm]{Images/mergestep1.png} }}%
		\qquad
		\qquad
		\subfloat[A and B merges. A third node appears]{{\includegraphics[width=5cm]{Images/mergestep2.png} }}%

		\subfloat[All nodes joined in transient group]{{\includegraphics[width=5cm]{Images/mergestep3.png} }}%
		\qquad
		\qquad
		\subfloat[Splitting to reduce group size]{{\includegraphics[width=5cm]{Images/mergestep4.png} }}%
		\caption{Simplified illustration of the idea behind splitting. K = 2}%
		\label{fig:splitting}%
\end{figure}



\subsection{Introduction to K-means}
K-means is a clustering algorithm commonly used for classification purposes in machine learning, usually in the context of unsupervised learning. K-means
is an iterative algorithm designed to find a predefined amount of clusters in a data-set with unclassified data points. The amount of clusters to find is denoted by
K (hence K-means). 

K centroids are randomly placed somewhere in the data set, then the distance between each data point and each centroid is computed. All data points
closest to the same centroid becomes part of the same cluster set. When all data points are associated with a cluster, each cluster computes the mean position of all its data points.
The value of the mean position becomes the location of the new centroids. When all the new centroids are computed, one iteration is over. The algorithm continues until the position 
of all centroids remains unchanged after an iteration. When the position of the centroids are the same two iterations in a row, it means that the solution has converged and all K-clusters has been identified \cite{mackaymeans}.

A well known problem with K-means is that the result strongly depends on how the algorithm is initiatied. This problem can be reduced by running K-means several times
and choosing the best result of all the runs. 

\subsection{K-means splitting}
K-means clustering is not directly applicable to solve the clustering challenge in this thesis. The simple reason being the distributed nature of
the nodes and groups. Indeed, a vanilla implementation of K-means requires an extensive overview of the surronding, if not entire, network topology.
The groups are so far limited to knowing about their members and their neighbours. It would also be hard to choose a suitable K. 

Nontheless, K-means could still potentially help us create better groups. Let us consider the following.
By extending the K-Closest Neighbour Clustering so groups are allowed to merge into a transient group if it exceeds the given maxsize.
The purpose of the large transient group is to identify the biggest gap between nodes, and split the group in two new groups with
$count(nodes) < maxsize$. If we set the K-means variable K, to $K=2$, K-means can be used to identify where the transient group should be split to create two more connected clusters. 

Randomly picking two nodes to be centroids could lead to undesirable results where the two resulting groups are fundamentally different from the original ones.
Recall that the purpose of the split is to reevaluate the divison line between the groups and see if K-means can achieve a better split, not create two entirely new clusters.
There already exists two independent clusters before the split. The centroids of these clusters can be precalculated before merging, and be used to bootstrap K-means with two non-randomly chosen
centroids.
If the distance between the most interfering nodes is lower after applying the K-means split, the new groups are applied. On the other hand, if the distance is shorter
(meaning a less optimal cut is found) the original groups are restored. As the bootstrapped position of the centroids is planned and not random,
the result is deterministic and the algorithm does not have to be run more than one time to get the result. 


\subsection{Results}
The results of simulating the group creation with K-means splitting implemented can be seen in figure \ref{fig:kmeans} (see 
appendix \ref{appendix:kmeanssplit} for full size images). The groups clearly look very similar to those we computed in \ref{fig:knearest}, but in some places there are some major differenes. This is where splits have happened. A comparison of the two algorithms executed on the same section of Tynset's town centre can be seen in figure \ref{fig:kmeanscomparison}, and another
comparison of a section of the uniform distribution can be seen in figure \ref{fig:kmeanscomparisonuniform}. 

\begin{figure}
	\centering
		\subfloat[Uniform]{{\includegraphics[width=5cm]{Images/computations/KMEANS500x500_1000n.jpg} }}%
		\qquad
		\qquad
		\subfloat[Forks]{{\includegraphics[width=5cm]{Images/computations/KMEANSForks.jpg} }}%

		\subfloat[Lillehammer]{{\includegraphics[width=5cm]{Images/computations/KMEANSLillehammer.jpg} }}%
		\qquad
		\qquad
		\subfloat[Tynset]{{\includegraphics[width=5cm]{Images/computations/KMEANSTynset.jpg} }}%
		\caption{K-means splitting on different topologies}%
		\label{fig:kmeans}%
\end{figure}

\begin{figure}
	\centering
		\subfloat[Without K-means splitting]{{\includegraphics[width=2.5cm]{Images/computations/TynsetNear.jpg} }}%
		\qquad
		\subfloat[With K-means splitting]{{\includegraphics[width=2.5cm]{Images/computations/TynsetNearKmeans.jpg} }}%
		\caption{Comparison with and without K-means splitting on Tynset}%
		\label{fig:kmeanscomparison}%
\end{figure}

\subsection{Evaluation}
There is little doubt that K-means splitting improves upon the original algorithm, especially when splits only are accepted when the maximum interference of a group is reduced after the split.
Even though the clusters in a static environment is slightly improved, the most important use-case for K-means splitting is under the introduction of new nodes to an environment
where all groups have converged. Where in the unmodified K-Closest Neighbour Clustering algorithm there was no specific way to handle a new node in an environment of saturated groups
with K-means splitting eventually new groups will be formed on the basis of the updated topology. The weakness of this approach is how ill suited it is to run in a distributed environment.
It is trivial to simulate splits using K-means, as during a simulation the absolute coordinates of the nodes in the group are accesible. This is information unavailable for nodes under
real circumstances - the only available information is the approximiate point-to-point distance inferred from the signal strength scans.
Hence some major adaptions would have to be made to get this to work in an implementation.
One way to realize the algorithm would be to select a physical node as centroid instead of a virtual point. All nodes in sensing range of the surrounding area of
the centroid node would report the distance to the centroid to their n-hop neighbour using an overlay network to communicate internally in the group. The other nodes could
then figure out their approximate distance to the centroid. It is unknown if such an approach would be accurate enough to work, or have a complexity that is conceivable to implement.  

\begin{figure}
	\centering
		\subfloat[Without K-means splitting]{{\includegraphics[width=4cm]{Images/computations/500x5000Near.jpg} }}%
		\qquad
		\subfloat[With K-means splitting]{{\includegraphics[width=4cm]{Images/computations/500x5000NearKmeans.jpg} }}%
		\caption{Comparison with and without K-means splitting on uniform distribution}%
		\label{fig:kmeanscomparisonuniform}%
\end{figure}


\section{Minimum Cut Splitting}
In this section we will consider another approach to group splitting, namely the graph theory algorithm for finding a minimum cut in a directed graph. 

\subsection{Minimum cut}
The core concept of the minimum cut algorithm is to find a way to cut a directed graph so that a source node is completely separated from a sink node, while cutting over edges 
whose weights adds up to a sum which is as low, or minimal, as possible \cite{chinneck}.

The minimum cut is often referenced in the context of the max-flow min-cut theorem, which states that the maximum flow of a directed graph is equal to the capacity of the minimum cut.
The capacity of the minimum cut is equal to the sum of the weight of the edges the cut runs through. In 1956 Delbert R. Fuelkerson and Lester R.
Ford proposed a method for finding the minimum-cut \cite{ford1956} in a flow network, now called the Ford-Fuelkerson method. The method was extended by the Edmonds-Karp algorithm \cite{Edmonds} which is identical, except that is specifies that a breadth-first search should be used for the augmented-path stages in the algorithm, whereas in Ford-Fuelkerson this is unspecified (hence being called
a method, and not an algorithm).

The implementation of minimum cut that will be used in this section is a part of the NetworkX \cite{NetworkX} python package for manipulation of complex networks.

\subsection{Using minimum cut for group splitting}
Minimum cut can be used for group splitting by identifying the least connected partitions of a group. If a node, or a partition of the group, is connected to the rest of the group
through one or more links which are weaker than the strongest link to a neighbouring group, the partition can be excluded from the group to allow for a merge with the neighbouring group. 


\begin{figure}
	\includegraphics[width=\textwidth]{Images/mincutflow2.png}
		\caption{Flowchart of the minimum cut implementation for splitting}%
		\label{fig:mincutflow}%
\end{figure}


	\begin{figure}
		\tiny
		\begin{python}
			def findCutAndPartition(G, sink):
			minCutCapacity = 99999 #Initiated to infinite (or a very high number)
			minCutPartition = [] #Partition initially empty
			for source in G.nodes_iter():
				if source == sink: #Source and sink should not be the same
						continue
				cut, partition = nx.minimum_cut(G, source, sink)
				if (cut >= 99999): #If cut is too high (infinite), dont keep
						continue
				if cut < minCutCapacity:
						minCutCapacity = cut;
						if sink in partition[0]:
								minCutPartition = partition[1]
						elif sink in partition[1]:
								minCutParttition = partition[0]

			for n in minCutPartition: #Remove nodes from graph
						G.remove_node(n)
			return list(minCutPartition);
		\end{python}
		\caption{Pseudocode of the minimum cut stage of the splitting}
		\label{fig:pseudocut}
	\end{figure}


To utilize minimum cut as a mechanism for splitting, the groups has to be treated as directed graphs. While the data structure of the networks in the implementation is not
originally designed for graph operations, the nature of the nodes with its neighbour lists and signal strength metrics certainly can be treated as a graph. Meaning that each node's neighbour list contains
its outgoing edges, while the belonging signal strength value is the weight of the edge. In contrast to the K-means splitting implementation, the minimum cut implementation does not require
a larger transient group. Instead the groups have to negotiate to know when a split can be confirmed. A flowchart illustration an abstract flow of the simulation implementation can be seen in figure \ref{fig:mincutflow}, showing the steps which are more closely described below. 

\begin{enumerate}
	\item Two groups $A$ and $B$ wants to merge, their combined size is larger than the specified group $maxSize$. The measured maximum value of signal strength between nodes in the two different
		groups $A$ and $B$ is called $rssiThresh$. 
	\item The groups independently build a graph of their current state, using the signal strength values as weights on edges to neighbouring nodes. In the simulation implementation this is done using the NetworkX graph.
	\item Each edge in the graph where the weight is larger than  $rssiThresh$, gets a new weight = \verb|INFINITE|.
	\item The groups independently perform minimum cut multiple times. In the simulation implementation this is done with a call to the library function \verb|minimum_cut(G, s, t)|,
		where $G$ is the graph, $s$ source and $t$ is the sink. To find the best partition to exclude from the graph, the  sink node is always the node which lies closest to the other group,
		while the source node changes for each call to minimum cut. When all nodes, except for the sink, have been sources in the minimum cut algorithm, this step is finished. The cut
		which has the lowest capacity of all performed cuts is stored along with the partition of this cut. 
	\item The partition which does not contain the sink node is excluded from the graph, however if this partition is empty it means there exists no suitable cuts in the group which are less beneficial than the addition of another group. 
	\item The groups $A$ and $B$ check if the combined number of nodes exceeds $maxSize$. If the node number is still too high, repeat step 4 provided that there were any 
		exclusions in the graphs in the previous run. If there are no exclusions in either group, there exists no cut that can bring the combined group size down to desired size, and the
		merge has to be aborted. 
\end{enumerate}

The pseudocode for step 4-5, referred to as the \verb|findCutAndPartition| procedure in the flowchart of \ref{fig:mincutflow}, can be seen in \ref{fig:pseudocut}.  

For a detailed step by step illustration describing how a split happens in a topology where a split would be beneficial, see figure \ref{fig:mincutstep}. 
\begin{figure}
	\centering
		\subfloat[Two groups are initially too large to merge. A split is going to be attempted. ]{{\includegraphics[width=7cm]{Images/mincutstep1.png} }}%
		\qquad
		\subfloat[Both groups identify edges which should not be cut, marked by an infinite edge weight]{{\includegraphics[width=7cm]{Images/mincutstep2.png} }}%
		
		\subfloat[The groups excludes the partitions with the lowest capacity cuts]{{\includegraphics[width=7cm]{Images/mincutstep3.png} }}%
		\qquad
		\subfloat[The number of members of the groups combined is now lower than maximum size, and the merge is succesful]{{\includegraphics[width=7cm]{Images/mincutstep4.png} }}%
		\caption{Minimum cut illustration with group maximum size set to 5}%
		\label{fig:mincutstep}%
\end{figure}


\begin{figure}
	\centering
		\subfloat[Uniform]{{\includegraphics[width=4.5cm]{Images/computations/MINCUT500_500.png} }}%
		\qquad
		\qquad
		\subfloat[Forks]{{\includegraphics[width=4.5cm]{Images/computations/MINCUT_FORKS.png} }}%

		\subfloat[Lillehammer]{{\includegraphics[width=4.5cm]{Images/computations/MINCUT_LILLEHAMMER.png} }}%
		\qquad
		\qquad	
		\subfloat[Tynset]{{\includegraphics[width=4.5cm]{Images/computations/MINCUT_TYNSET.png} }}%
		\caption{Minimum cut algorithm splitting on different topologies}%
		\label{fig:mincutresults}%
\end{figure}


\subsection{Simulation results}
The results of the simulation, using the same four topologies used for previous simulations, with the implementation of minimum cut splitting can be seen in figure \ref{fig:mincutresults} (full scale images can been seen in \ref{appendix:mincutsplitone}).
Different groups are distinguished by color, and the group max size is set to 128. 

\subsubsection{Comparison}
In the result section of K-means splitting there was a close up examination and comparison of the group division in Tynset's city centre using no splitting and K-means splitting.
In figure \ref{fig:mincutcomparison} a close up of the same area has been made between K-means splitting and minimum cut splitting. The group division seems to 
be very similar, except for a major difference in a cluster on the top left area of the topology (pink color on \ref{fig:mincutcomparison}a).
While the K-means splitting algorithm identifies a neatly separated cluster, the same cluster in the minimum cut method does not include three nodes at the bottom right edge of the cluster.
This is imperfect  behaviour. By looking at the minimum cut, figure \ref{fig:mincutcomparison}b, one can tell that the distance from the three pink nodes to the pink cluster is large.
Actually, it is larger than the distance to the green cluster. The green cluster is far from being the max size of 128, so upon a merge, shouldn't a split happen which would redefine the
groups in a better way? What could possibly be the cause of this less optimal behaviour?
\begin{figure}
	\centering
		\subfloat[With K-means splitting]{{\includegraphics[width=3cm]{Images/computations/TynsetNearKmeans.jpg} }}%
		\qquad
		\subfloat[With minimum cut splitting]{{\includegraphics[width=3cm]{Images/computations/TynsetNearMincut.jpg} }}%
		\caption{Comparison between K-means splitting and minimum cut splitting on Tynset}%
		\label{fig:mincutcomparison}%
\end{figure}



\subsubsection{Algorithm inspection}
The application created earlier to step through each iteration of the algorithm can help illuminate the problem.
The entire solution converges in 11 iterations, but iteration 8 and 9 seems to be the iterations which spefically impacts the result of the cluster in question.
These are illustrated in figure \ref{fig:mincutbug}. In \ref{fig:mincutbug}a, the green group seeks to merge with the pink. They lie very close with a signal strength between them being -45 dBi.
During the split several nodes are kicked out, which can be seen in \ref{fig:mincutbug}b. As expected, the three nodes in question have not been kicked out. By watching the figure it is easier to understand why.
These 3 nodes are obviously less tighter connected than -45dbi, which would the threshold for minimum cuts over this graph, but they are not kicked out because sometime during the split the size of the group becomes lower than the max size of $128$ - before they have the chance to be thrown out. The merge is then immediately accepted. 

An improvement to the algorithm would be to change step 6, so that the minimum cut partitioning would run until no more nodes could be kicked out. Even if the group size reached a tolerable size before that. A change
like that would prevent leaving nodes in a group when their neighbours have been kicked out, and they would also be better off in another. 

\subsubsection{Results after re-evaluation}
The algoritm results after the modification is certainly more satisfactory than before. The final comparison between K-means and the new minimum cut can be seen in figure \ref{fig:mincutbug}. For the full size version of
all four topologies on this simulation as well, see appendix \ref{appendix:mincutsplittwo}. 

\begin{figure}
	\centering
		\subfloat[Iteration 8: The green group seeks to merge with the pink, the signal strength between them for instance being -45dBi]{{\includegraphics[width=4cm]{Images/mincutbug1.png} }}%
		\qquad
		\subfloat[Iteration 9: After the minimum cut split, some nodes have been thrown out to allow for the merge]{{\includegraphics[width=4cm]{Images/mincutbug2.png} }}%
		\caption{Illustrating two algorithm iterations to find the problem source of the minimum cut}%
		\label{fig:mincutbug}%
\end{figure}




\subsection{Evaluation}
While being slightly more complex to implement in a simulation than the K-means splitting, the results are more reliable and less dependent on 
initial group selection. In all simulations the groups have been merged iteratively, meaning that the merge sequence depends
on the order of merges. With this minimum cut implementation the resulting groups will become the same, no matter the order
of group merging. This is a property which can be considered of importance in a real world scenario. The major benefit 
the minimum cut group splitting has over K-means splitting in a real-world implementation scenario is the need for only link weights.
As mentioned, K-means would be hard to realize, while the minimum cut method should in theory be little different from simulation 
to real world implementation. 


\section{Evaluation of the clustering study}
In this chapter four methods of clustering has been considered: regular agglomerative clustering, K-Closest Neighbour Clustering (KCN), KCN Clustering with K-means splitting
and KCN Clustering with minimum cut splitting. Which of the first is a well known clustering algorithm, and the last three has been proposed in this thesis. 
This section is dedicated to the evaluation of the work done throughout this chapter, with regards to gained knowledge and simulation weaknesses.

\subsection{Knowledge acquired from simulations}
Before beginning the work on the thesis it was hard to envision how groups could be formed in a distributed manner, without any central controller having a top down view of
the landscape of wireless access points. While many questions still remain, some of which will be addressed in the next chapter, many has also been answered. The main issue
addressed is how groups can be formed, and how they can be continously updated to keep up with changes in the surrounding network.  

Unsurprisingly, the small modification to the agglomerative clustering algoritm, which made KCN Clustering, yielded promising results.
This algorithm could provide a natural starting point for a real-world implementation in a static network topology.
However, this specific algorithm does not handle changes in the network environment very well, as once a group's maximum size is reached, it can not be modified. 
Still it could possibly provide useful for testing scenarios, maybe in an early implementation when a distributed channel allocation algorithm is to be tested across the group, without
wanting changes to happen. By considering ways to improve this algorithm, the notion of splitting was introduced, and tested through the methods of K-means and minimum cut.

K-means was an interesting clustering algorithm to work with. The promising splitting results observed on the simulation topologies was especially fun to see, as
the K-means method was adapted to fit the clustering purpose, and not implemented as a textbook K-means clustering.
However, when comparing the K-means and the minimum cut algorithm to decide which one is the best bet for a real-world implementation of a splitting algorithm,
the minimum cut algorithm comes off as the easiest to implement given the limited knowledge of the surrounding nodes each access point has access to. 
Minimum cut also has the benefit of being more predictable in its splitting. The layout of a network is so similar to a graph,
that in hindsight a version of the minimum cut algorithm seems like such an obvious choice. 

\subsection{Simulation weaknesses and data bias}
As seen the simulations have taught us about what kind of clustering could work under the assumptions made early in the thesis. The obtained results are also 
a product of the data that was gathered, and the simulation program which was made. This subsection will consider all the different aspects that has not been considered
when producing the simulated results, but might have an impact in a real life implementation. 

\subsubsection{Volatility} 
All the simulations takes a static network topology as an input. The static topology is an image of a network topology in a given state, and is not representative
of the volatile nature of modern networks. Actually, it is more like a computation than a simulation in its rightful meaning.
While it is true that most access points are relatively stationary and constant, it is increasingly popular to create ad-hoc on-demand Wi-Fi
networks using cell phones or a computers as access points. Additionally resetting of routers, power outages, etc. means that access points come and go. 
Hence it is indisputable that network topologies are volatile, and the group clustering algorithm should, when final, run as a continous operation. 
Any reader of this thesis will know that ways the algorithm can handle volatily has been considered, enabled by group splitting,
but the simulation framework does not support inserting or removing nodes during the computation, hence only static environments that quickly converges have been tested. 

\subsubsection{Signal strength reporting bias}
The group computations has been based on the signal strength of all nearby access points, and the $dBi$ values that denotes all signal strengths has been calculated using the
free space path loss formula. As distance is the only variable that affects the result of the signal strength calculation,
it means that the signal strength levels becomes a symmetric binary relation between nodes. In the real world this is not the case, and also the reason allgomerative clustering was deemed
less ideal. Access points may interfere with each other differently, and the signal strength values may change when there are variations in the environment. This is the reasons deadlocks could
occur if implementing regular agglomerative hierarchical clustering and the motivation behind creating K-Nearest Neighbour Clustering, even though the topologies actually
could work well with agglomerative clustering. We also have no simulations of how groups would look if the perceived signal strength between two nodes were not mutual.

\subsubsection{Dimensionality and node location reporting bias} 
Both the data that has been randomly generated, and the data fetched from \verb|wigle.net| takes the $x$ and $y$ axes in account. Obtaining data in three dimensions is harder.
Wigle places the access points using longitude and latitude as all their data is acquired using ground-level triangulation
By creating clusters based on information in two dimensions there is no way to know how the algorithm behaves in a 3-dimensional world. Of course, it is reasonable to think that an algorithm
which is based on distance two dimensions should also work well in three dimensions, but there has been done no simulations in three dimensions. As mentioned in the data-mining chapter, Wigle has a 
tendency to place nodes closer to roads, which means that the positioning of nodes may not be strictly accurate in two dimensions either. 



%\subsection{Uniformly distributed nodes}
%We will look at how groups were created in different topology scenarios. 
%All topologies presented in this section was created by the topology generation program,
%but with different input parameters. Groups are distinguished by node color, where nodes
%of the same color represents members of the same group. 
%%\subsubsection{Scenario 1}
%Computed with 200 nodes with a maximum of 128 members in each group.
%
%As can be seen in figure \ref{fig:200_128} the algorithm divides the nodes in two
%sections. For clarity, a divisive line has been drawn around each group,
%in case colors are not available.
%When two major groups merged, the biggest groups surpasssed 128 members and began
%kicking out members. The excess members formed the black group at the bottom. 
%
%\begin{figure}
%\center
%\includegraphics[scale=0.45]{Images/grouptest_1.jpg}
%\caption{200 nodes, $memberThresh=128$, $size=200x200$
%\label{fig:200_128}
%\end{figure}
%
%
%\subsubsection{Scenario 2} \label{scen2}
%Computed with 200 nodes with a maximum of 10 members in each group.
%
%The result of this computation, seen in figure \ref{fig:200_10}, is a little less obvious.
%The groups are again distinguished by different color, but for clarity we add a gray connecting
%blob for nodes in the same group. Also blobs connected with a line are in the same group. 
%
%It is worthy to take notice that one group is especially scattered around the graph.
%At first eyesight, it looks like an algorithm deficiency, but the reason is quite simple:
%when nodes are kicked out of a group during a merge, they will connet to other nodes
%that belong in a group where $n$ has not yet reached $memberThresh$. When this have happened a couple of times, everyone has found a group except for the remaining few.
%These are typically straggler nodes or smaller clusters separated from the others.
%They are not big enough to reach the group $memberThresh$ on their own, so the merge with other
%nodes that are in unmaxed groups. Thus, even though they have neighbours which
%influence them more, they can only merge with nodes further away,
%because that is the only unlocked group that remains.
%\begin{figure}
%\center
%\includegraphics[scale=0.45]{Images/grouptest_2.jpg}
%\caption{200 nodes, $memberThresh=10$, $size=200x200$}
%\label{fig:200_10}
%\end{figure}
%
%\subsubsection{Scenario 3}
%Computed with 5000 nodes, with a maximum for 64 members in each group. 
%
%Figure \ref{fig:2000_64} shows the result of the computation. Because of the quantity
%of nodes and the clear separation of groups, they are easily distinguished by color.
%This topology is much denser than the others, and can vaguely resemble the density of highly populated areas.
%
%We can clearly see that the overall tendency is that groups are formed
%in concentrated areas of nodes. However, some groups are scattered, sometimes
%all over the map. An example of a scattered group is highlighted
%in figure \ref{fig:2000_64}. Its  member nodes has a thicker black line around them. 
%\begin{figure}
%\center
%\includegraphics[scale=0.45]{Images/scenario3alt.png}
%\caption{5000 nodes, $memberThresh=64$, $size=2000x2000$}
%\label{fig:2000_64}
%\end{figure}






%
%\begin{figure}
%\centering
%\begin{minipage}{.6\textwidth}
%	\center
%	\includegraphics[width=0.9\linewidth]{Images/grouptest_1.jpg}
%	\captionof{figure}{\newline200 nodes, $memberThresh=128$}
%	\label{fig:200_128}
%\end{minipage}%
%\begin{minipage}{.6\textwidth}
%	\center
%	\includegraphics[width=0.9\linewidth]{Images/grouptest_2.jpg}
%	\captionof{figure}{Another figure}
%	\label{fig:test2}
%\end{minipage}
%\end{figure}



%
%\section{Results}
%By running the scripts that parses data from Wigle on populated areas, we should get an idea 
%on how the algorithm performs in more realistic topologies. We will have a look at three
%scenarios where the group allocation data is based on AP-data. 
%
%Three suitable locations has been selected to perform the testing on Lillehammer (Norway)
%a smaller city, Tynset (Norway) a less densely populated area, 
%and Forks (Washington, United States). All tests were 
%ran with a maximum group size of 128, and a $-dBi$ threshold of $-80$. 
%
%\subsubsection{Lillehammer}
%\begin{figure}
%\center
%\includegraphics[scale=0.46]{Images/cities/lillehammer_groups.jpg}
%\caption{Lillehammer}
%\label{fig:lillehammer_topo}
%\end{figure}
%
%The computation results of Lillehammer can be seen in figure \ref{fig:lillehammer_topo}.
%The topology is of medium density, consisting of 4990 APs, and is 2572 meters high 
%and 8418 meters wide. From the tight clusters in the middle it is easy to make out the city centre.
%We can clearly see different groups with different sizes. Some of the smallest groups
%are highly likely so small because the distance to other nodes is too high for
%the group to hear. In the denser areas they are occasionally very
%entangeled, and it can be hard to make out the group borders.
%
%Another thing to notice is that APs are nearly always placed in straight lines.
%The straight lines are roads, and as Wigle collects data based on triangulation, the nodes
%that is only seen once will get the position they are observed in, and not an actual
%triangulated position. 
%\subsubsection{Tynset}
%The computation results of Tynset can be seen in figure \ref{fig:tynset_topo}. 
%The topology consists of 726 APs, is 1670 meters high and  6720 meters wide. 
%Unsurprisingly it resembles Lillehammer on a smaller scale.
%Again we see
%some very clearly defined groups, but in the city centre there are groups
%which overlaps. We can also see nodes that are alone in their group,
%because they are too far away from anyone else.
%Much like Lillehammer, this topology is also strongly affected by the weak
%
%triangulation of the APs, so most APs seems to be placed on top of a road. 
%
%\begin{figure}
%\center
%\includegraphics[scale=0.46]{Images/cities/tynset_groups.jpg}
%\caption{Tynset}
%\label{fig:tynset_topo}
%\end{figure}
%
%\subsubsection{Forks}
%
%The computation results of Forks can be seen in figure \ref{fig:forks_topo}. 
%The topology consists of 1715 nodes, and is 2122 meters high and 4495 meters wide. It
%is important to include, because it is quite different from the other topologies and
%represents a variation from the typical town and city structure of Norway.
%The size of the groups are a little more uniform when comparing it to the others.
%This can explained by the the smaller area the town is contained within. When a group
%is not full, it will almost always hear someone that it can merge with.
%We still have groups overlapping each other in the denser regions in Forks as well. 
%What is worth noticing is that the APs are positioned more realistically as locations
%of households. American towns looks more like a grid with roads in between households,
%which makes triangulation easy and a lot more accurate. 
%
%\begin{figure}
%\center
%\includegraphics[scale=0.46]{Images/cities/forks_groups.jpg}
%\caption{Forks}
%\label{fig:forks_topo}
%\end{figure}
%
%
%
%

\chapter{Future Work}
\section{ResFi}
\section{Raft}


%
%\makeatletter
%\def\BState{\State\hskip-\ALG@thistlm}
%\makeatother
%\subsection{Pseudocode}
%\begin{algorithm}
	%\caption{Connected group}\label{congroup}
	%\begin{algorithmic}[1]
	%\Procedure{MergeGroup}{}
	%		\State $\textit{members} \gets \text{1}$
	%		\State $i \gets \textit{patlen}$
	%		\If {$i > \textit{stringlen}$} \Return false
	%		\EndIf
	%		\State $j \gets \textit{patlen}$
%			\If {$\textit{string}(i) = \textit{path}(j)$}
%			\State $j \gets j-1$.
%			\State $i \gets i-1$.
%			\EndIf
%			\State $i \gets i+\max(\textit{delta}_1(\textit{string}(i)),\textit{delta}_2(j))$.
%			\State \textbf{goto} \emph{top}.
%		\EndProcedure
%	\end{algorithmic}
%\end{algorithm}


\backmatter
\printbibliography

\begin{appendices}
\chapter{Data generation}
\subsection{GenerateTopology.py}
\begin{python}

class Node:
	_ssid = None
	_channel = None
	_neighbours = None
	gg_neighbourMembers = None
	_freq = 0
	_constant  = -27.55
	_minimumInterference = 0
	group = None
	name = None
	x = 0
	y = 0

	def __init__(self, posx, posy, n, freq, thresh, name = None):
		self._neighbours = []
		#Data structure 
		self.edges = {}
		self._minimumInterference = thresh*-1
		self.x = posx
		self.y = posy
		self._freq = freq
		if (name == None):
			self.name = "NODE" + str(n)
		else:
			self.name = name
		self._ssid = self.name

	def __hash__(self):
		return hash(self.name)

	def rssiNeighbour(self, node):
		for n in self._neighbours:
			if n["ssid"] == node.name:
			return n["dbi"]
		return None

	def distanceTo(self, node):
		x = node.x - self.x
		y = node.y - self.y
		return math.sqrt(x**2+y**2)

	def getNeighbourCount(self):
		return len(self._neighbours)

	def getMostDisturbing(self):
		highest = -100
		nodeinfo = None
		for n in self._neighbours:
			if n["dbi"] > highest and n["obj"].group.name != self.group.name and not n["obj"].group.locked:
			nodeinfo = n
			highest = n["dbi"]
		return nodeinfo

	def getLeastDisturbingCompanion(self):
		lowest = 100
		nodeinfo = None
		for n in self._neighbours:
			if n["dbi"] < lowest and n["obj"].group.name == self.group.name:
			lowest = n["dbi"]
			nodeinfo = n
		return nodeinfo

	def getDBSum(self):
		sum = 0;
		for n in self._neighbours:
			sum += n["dbi"]
		return sum

	def calculateInterferenceTo(self, nodeObject):
		if self == nodeObject:
			return
		dist = round(self.distanceTo(nodeObject))
		if (dist == 0):
			dBi = -40
		else:
			dBi  = self.measureDbi(dist)*-1
		if (dBi > self._minimumInterference):
			self._neighbours.append({"ssid": nodeObject._ssid, "dbi": dBi, "obj:": nodeObject})

	def getData(self):
		data = OrderedDict()
		data.update(posX = self.x)
		data.update(posY = self.y)
		data.update(frequency = self._freq)
		data.update(ssid = self.name)
		data.update(neighbourCount = len(self._neighbours))
		data.update(neighbours = {})
		for i in range(len(self._neighbours)):
			data["neighbours"].update({i : {"ssid" : self._neighbours[i]["ssid"],
																			"dbi" : self._neighbours[i]["dbi"]}})
		return data

	def measureDbi(self, dist):
		return (20 * math.log(self._freq, 10)) + (20 * math.log(dist, 10)) + self._constant

	def __str__(self):
		return self.name

	def __repr__(self):
		return self.name

	def __unicode__(self):
		return self.name

class Topology: 
 _map = {}
 _width = None
 _height = None
 _spacing = None
 _nodes = []
 _nodesDict = {}
 _nodeCount = 0
 _frequency = 2437
 _thresh = 0
 _minimumRadiusVectors = None
 premadeNodes = None

 def __init__(self, width, height, spacing, nodeCount, dbThresh, premadeNodes = None):
	self._thresh = dbThresh
	self._width = width
	self._height = height
	self._spacing = spacing
	self.premadeNodes = premadeNodes
	if (premadeNodes != None):
		self._nodeCount = len(premadeNodes)
	else:
		self._nodeCount = nodeCount

 def newTopology(self):
	self.createMinimumRadiusVectors()
	self.populateMap()
	self.measureInterference()

 def getNodeCount(self):
	return self._nodeCount

 def getNodes(self):
	return self._nodes;

 def initMap(self):
	print("Generated topo")
	print("Generated filled topo with None objects")

 def measureInterference(self):
	print("> Calculating interference between all nodes.")
	i = 0
	printPercentage = 5
	percentageMark = len(self._nodes) / (100 / 5)

	for nodeSubject in self._nodes:  
		if (i % percentageMark == 0):
		print(" *", i, "of", len(self._nodes), "nodes done.")

		for nodeObject in self._nodes:
		nodeSubject.calculateInterferenceTo(nodeObject) 
		i += 1

 def createNode(self, posx, posy, nodeNumber, nodeFreq, nodeDbiThresh, name=None):
	node = Node(posx, posy, nodeNumber, nodeFreq, nodeDbiThresh, name=name)
	try: 
		self._map[posy][posx] = node
	except KeyError:
		self._map[posy] = {}
		self._map[posy][posx] = node

	self._nodes.append(node)

 def generateRandomNodes(self):
	nodeCount = 0
	for i in range(self._nodeCount): 
		while 1:
		y = random.randint(0, self._height - 1)
		x = random.randint(0, self._width - 1)
		if self.isPositionAvailable(x, y) == True:
			self.createNode(x, y, nodeCount, self._frequency, self._thresh)
			nodeCount += 1
			break

 def placeExistingNodes(self):
	nodeCount = 0
	print("> Placing nodes in topology.")
	for n in self.premadeNodes:
		self.createNode(n['x'], n['y'], nodeCount, self._frequency, self._thresh)
		nodeCount += 1
	print("> Nodes placed.")

 def populateMap(self): 
	if self.premadeNodes == None:
		self.generateRandomNodes()
	else:
		self.placeExistingNodes()

 def isPositionAvailable(self, testx, testy):
	for pos in self._minimumRadiusVectors:
		x = testx + pos[0]
		y = testy + pos[1]
		node = None
		try:
		node = self._map[y][x]
		except KeyError:
		return True
	return False

 def createMinimumRadiusVectors(self):
	"""Creates a list of relative positions to a node, where
	no other node can be placed because of the minimum spacing
	between nodes."""
	positions = []
	for i in range(-self._spacing, self._spacing + 1):
		for j in range(-self._spacing, self._spacing + 1): 
		dist = math.sqrt(i**2+j**2)
		if dist <=  self._spacing:
			positions.append((i, j))
	self._minimumRadiusVectors = positions

 def printTopology(self):
	for y in range(len(self._map)):
		print(self._map[y])

 def writeData(self, outfile):
	data  = OrderedDict()
	data.update(mapWidth = self._width)
	data.update(mapHeight = self._height)
	data.update(nodeCount = self._nodeCount)

	allnodes = {}
	for i in range(self._nodeCount):
		allnodes.update({i : self._nodes[i].getData()})

	data.update(nodes = allnodes)
	j = json.dumps(data, indent=2) 
	f = open(outfile, "w")
	f.write(j)
	f.close()
	print("> Topology data written to file: ", outfile)

def main(): 
 args = parseOptions()
 print("Width: ", args.width, " height: ", args.height, "spacing: ", args.spacing)
 topo = Topology(args.width, args.height, args.spacing, args.nodes, args.thresh)
 topo.newTopology()
 topo.writeData(args.output)



\end{python}
\subsection{WigleData.py}

\end{appendices}




\end{document}
