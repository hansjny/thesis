\chapter{Survey of related work}
This chapter is dedicated to give a brief account of works related to controlling and managing 802.11 wireless access points. 

\section{Cisco RRM}
\subsubsection{Description}
CISCO offers a RRM (Radio Resource Management) solution directed at enterprise networks \cite{ciscoRRM}.
Their architecture consists of one or more Wireless Lan Controllers (WLCs), which performs the responsibilities of regular APs, 
but also performs RRM, through grouping, leader election, dynamic channel assignment and transmit power control. Regular APs that 
are not WLCs, tunnel all their packets through the WLC unit they are registered with \cite{cisco_2009}. There are many finesses and details to the Cisco RRM, here we will only present the
major concepts.

\subsubsection{RF-Grouping}
In CISCO RRM, the RF-grouping stage is the identification of controllers and APs that are under the administrative control of the same RRM. 
To identify the RF-group an RF-group name is used, which is an ascii string sent to the APs from their preconfigured WLC. 
All APs broadcasts their RF-group name along with the IP-address of their controller.
Any other AP sharing the same RF-group name reports the incoming broadcast message to the controller, and then rebroadcasts the packet. 
This way the controller becomes aware of all APs that are in range of each other, similar to the flooding routing mechanism,
and the RF-group becomes aware of all the APs that can physically hear each other.
This is called the neighbour discovery stage, APs within physical hearing of -80dBi and above makes up what Cisco calls an RF-neighbourhood.
If applicable the RF-groups perform leader election to decide which controller becomes the RF Group leader, but a preconfigured leader can also be decided.
The RF group leader has the responsibility of running the relevant channel allocation algorithms and ajusting the radio power level of the APs. 

\subsubsection{Dynamic Channel Assignment}
To perform Dynamic Channel Assignment, the RF-leader has the neighbour lists for all access points in 
its RF-group. All APs track a set of metrics which tells something about its transmission quality. The metrics are: 
\begin{itemize}
	\item Co-channel interference from RF-group
	\item Co-channel interference from rogue APs (not a part of the group)
	\item Background noise that does not originate from 802.11
	\item Channel load measured at the physical layer
	\item Sensitivity to change (user selected variable to indicate how fast the network should respond to bad quality)
\end{itemize}
These metrics together end up in a unified metric called the Cost Metric, and the Cost Metric is then reported to the RF-leader. 
The DCA algorithm looks at the AP with the lowest cost metric first, meaning the AP with the worst QoS. The transmission channel can be changed for
this AP, and all 1-hop neighbours of this AP, but no AP further away than 1-hop. This is to prevent impacting APs throughout the RF-group.
That process is repeated, computing several channel plans, and at last, all the channel plans are evaluated through a cost function
to see if the Cost Metric has improved for the AP, without degrading the Cost Metric for neighbouring nodes. If an improvement is seen
then the channels are changed, and the algorithm moves on to the next AP on the list of low Cost Metrics.  

\subsubsection{Assessment}
It is valuable for us to understand the major mechanisms of Cisco RRM, as it is among the leading technologies for administrating larger amount of wireless access points. 
The RF-grouping stage with the neighbour discovery protocol are mechanisms that may be adaptable, or at least an inspiration for a future distributed protocol. The reason 
the Cisco RRM would not work well as it is in a chaotic, residential deployment may be quite obvious: the need for preconfigured WLCs and proprietary hardware does not fit well with the nomadic 
routers today, where house and apartment owners bring their own routers and access points. 

However, if an apartment building or a neighbourhood joined forces and invested in the Cisco RRM infrastructure, it would surely increase the experienced QoS. Should Wi-Fi be considered
a basic infrastructural requirement on the same level as water, sewage and electricity, then it might be better to have national Wi-Fi provider
that could ensure such large scale deployment of proprietary hardware. That's a digression, and maybe a discussion better left for politicians.



\section{DenseAP}
\subsubsection{Description}
DenseAP described in \cite{Murty2} aims to restructure the infrastructure of enterprise networks.  The two fundamental
changes they suggest is to deploy APs a lot denser (hence the name), and moving the task of associating clients
with APs to a centralized controller. The reasoning behind the dense deployment of APs is that signals diminishes
quickly in an indoor environment, and ideally a client should always be associated with an AP in the close vicinity. The argument 
they use for moving the association decision away from the client and over to a central controller, is that a client can
only use signal strength as the metric deciding which AP to associate with. This is emphasized as
suboptimal in conferences and meeting room environments, where many clients seek to associate with an AP at the same time.
If all clients pick the same AP it will naturally reduce the throughput on the medium because of the CSMA/CA protocol. 
Their infrastructure consists of DenseAP access points (DAPs) and DenseAP Controllers (DCs). The DAPs sends periodic
reports to the DC, which contains information as RSSI (Received Signal Strength Indication) measures, co-channel interference, and associated clients. Based on this
information the DC decides which DAP each client should associate with, and also which channel each DAP should transmit
on. 

\subsubsection{Assessment}
If we are to compare this solution with the Cisco RRM, the only thing in common is that this technology is also designed for enterprise networks to increase QoS over Wi-Fi. There
are very few other similarities, and while Cisco strives to optimize the channel plan and control of the transmission power of the antennas, DenseAP suggests having more APs, and control client association. As we are mainly interested in solutions that can work in a distributed, residential environment, there is very little this article and technology can help us with. 
On the contrary, the density of APs in apartment buildings are not too low, but too high, and there is usually no option to associate with another AP within one apartment.
This technology focuses less on optimizing the channel plan, and more on providing many different channels in the same area.  
To conclude, this technology is probably irrelevant for chaotic, residential deployments of Wi-Fi, and while it is an interesting approach to the problem, there is little of their
ideas that can be reused in this thesis. 


\section{HiveOS}
\subsubsection{Description}
HiveOs \cite{Aerohive}, developed by Aerohive Networks offers distributed protocols and mechanisms to improve Wireless LANs in enterprise networks. The APs 
in the network are called HiveAPs, and they offer services such as
\begin{itemize}
	\item Band steering: if an device can operate on the 5Ghz band, it will be forced to connect to the 5Ghz network to optimize the utilization of the radio spectrum. 
	\item Load balancing: all HiveAPs have real-time information about how clients performs. If a client tries to associate with a new HiveAP, it will only be accepted
				if the new HiveAP has a low enough load to handle more clients. It would also know if other neighbouring APs are better suited to handle the load of the new client.
				This is achieved by withholding probe responses.

	\item Channel allocation: by using the Aerohive Channel Selection Protocol, HiveAPs tries to select the channel with the lowest co-channel interference. Their channel selection protocol uses 5 measurements, two static and 3 dynamic.
	The first static measurements is the number of nearby APs who are operating on the same channel. The more APs there are the higher the penalty. The penalty per AP diminishes as the number of APs increases, this
		is because the first one(s) are the most critical. The other static cost factor is what power level can be transmitted at the given channel, as this may differ on some 5GHz non-overlapping channels. The dynamic measurements are CRC-error rate, channel utilization,
	and the utilization of overlapping APs. All of these dynamic factors can penalize a channel with 0 to 3.5%. 

\subsubseciton{Assessment}
\end{itemize}

\section{ResFi}
ResFi \cite{resfi} is the only significant related work that directly aims to enable self-organized management in residential deployments of wireless LAN. Works such as \cite{Murty2}, \cite{ciscoRRM} and \cite{Aerohive} are all directed toward enterprise
networks. As we also aim to mainly concern ourselves with residential networks and their infrastrucure, ResFi is especially interesting to look at. ResFi assumes that all access points have two interfaces, one connected to a wired backone (e.g Internet), and another
802.11 compatible wireless interface. The figures of ResFi includes a Radio Resource Management Unit (RRMU). This is simply the device that interfaces with the antenna, and in most residential homes it will be a router that controls the channel and the power levels of the antenna. In short ResFi enables communication between access points under different basic service sets without imposing a central controller on the access points or being a part of an extended service set. More importantly, ResFi enables all of this without doing any modifications to hardware and drivers (like modifying standards or requiring propriatary equipment). 

\subsection{Operation}
This section contains a short introduction of the way ResFi is initiated and operates.
\begin{enumerate}
	\item An AP that has just booted up beacons a frame on all available channels. This frame contains: a globally routable IP-address, port, and two cryptographic keys. One transient key for group communication, and the public key for the RRMU. 
	\item All APs that can hear the beacon responds with a probe response containing the same information as in 1.
	\item When the scan is complete, the new AP can establish secure point-to-point communicate with all other APs using the wired backbone, the globally routable IP-address and the cryptographic keys
\end{enumerate}

\subsection{Implementation}
This section is dedicated to a brief overview of how ResFi can be implemented and deployed.

Originally ResFi was implemented on Ubuntu 14.04. It adds vendor specific information elements to the MAC-header with the IP, port, and cryptographic data. This can be done Linux user space by using their modified version of hostapd \cite{resfigit}.
As it runs on python, it could in practice be implemented on any Linux system. It uses a south-bound API to communicate with the RRMU, and a north-bound API to enable applications to use ResFi. ResFi itself provides no specific channel allocation mechanism nor a group-creation/AP-clustering algorithm, but the north-bound API could fascilitate applications that provides services like these.

%\section{Distributed Clusteering}
%DCA and DMAC

\section{Channel allocation using DSATUR and SCIFI} 
This section is dedicated to previous work done using the DSATUR heuristic. A little note to this section: it would also fit well in the background section. It accounts for the fundamentally important priniciple that
channel allocation can be treated as a graph coloring problem once a network graph is constructed. As we wont be dealing with channel selection itself, the best fit is still in the related work section.  

\subsection{DSATUR}
DSATUR (from degree of saturation) is a heuristic created by Daniel Brélaz \cite{Brelaz} to find solutions to the NP-complete problem of coloring the vertices of a graph so that no adjacent vertices share the same color. 
Channel allocation schemes relying on the DSATUR algorithm has been proposed before. In 2004 a paper was published, called
"Automatic channel allocation for small wireless local area networks using graph colouring algorithm approach" \cite{mahonen}, where the idea is to listen for neighboring AP beacon frames to create a list of all neighbours.
This list would be broadcast to all the neighbours. For multi-hop support, all receiving nodes will rebroadcast the list in a fashion equal to the flooding routing mechanism. This enables routers to create a graph of access points, and the DSATUR algorithm can then be used to compute the channel distribution. A group like this is more or less exactly what we would like to achieve in this thesis, 
an we hope to begin resolving some of the problems this algorithm does not account for. In 2004, flooding the network to build a network graph to solve a NP-hard problem for, might have worked. Today this is not the case, as the size of the graph would be immense. 

\subsection{SCIFI}
SCIFI \cite{SCIFI} is a centralized channel allocation protocol for infrastructure Wireless LANs that improves the traditional graph-coloring algorithm DSATUR. 
While the paper shows that their central coordinator in fact improves the throughput compared to Wireless LANs that are not configured by SCIFI, the it is an algorithm for setting a channel in a preconfigured administrative domain. So, why is still mention here? Well, even though it does not deal with how the administrative domains
or AP clusters are defined, if the method of creating clusters of collaborating routers proposed in this thesis has merit,
SCIFI could plausibly be a supporting technology to compute channel distribution. Meaning one of our groups would act as the administrative domain required by SCIFI. 


