\chapter*{Acknowledgements}
There are many people I want to thank.\newline
First I would like to express my thanks and gratitude to my main supervisor Torleiv Maseng for the opportunity to work with this master thesis and his supervision the last two years. Without
your ideas, thoughts and feedback this thesis would not have been possible. \newline
Additionally I would like to thank Madeleine Kongshavn and Magnus Skjegstad for their participation in many phone meetings with me and Torleiv, and providing their thoughts and
input on difficult matters. \newline
I want to thank my co-supervisors Tor Skeie and Yan Zhang for agreeing to co-supervising this thesis and putting their knowledge at my disposal. 
\newline
Finally I would like to thank my girlfriend Martina Langseth Knutsen for her daily support and for taking the time to proofread this thesis.




\chapter*{Abstract}
Co-channel interference is the most significant degrader of quality of service in 802.11 Wi-Fi in dense residential areas.
The amount of observed neighbouring networks can grow as large as 20-30 networks in a modern apartment building. This makes
interference from neighbouring networks a big issue. It can degrade the perceived quality of service to the pointwhere private consumer networks can no longer provide
anywhere near the throughput guaranteed by the ISPs' service level agreements. This problem would be easier to solve if access points in residential areas had the
opportunity to cooperate and coordinate the distribution of channels in a way similar to what some centralized solutions for enterprise networks offer today.
This thesis is focused on developing a clustering algorithm that can define clusters of access points in a distributed and chaotic network topology. These clusters could 
provide a future framework for communication and coordination between access points.
We specify a set of requirements for the distributed clustering algorithm and then go through the development stages of the algorithm,
beginning with a minimal working algorithm, and iteratively progress until a satisfactory solution is found that meets the requirements.
Lastly, we suggest technologies that could facilitate the deployment of this clustering algorithm in a real world implementation on access points and consider
how these technologies could interface with each other to provide the required services.
