\documentclass[a4paper,UKenglish]{report}
\usepackage[utf8]{inputenc}
\usepackage{babel,csquotes,duomasterforside}
\usepackage[backend=biber,
	    style=numeric,
	    citestyle=numeric]{biblatex}

\addbibresource{references.bib}

\title{Group creation in a collaborative P2P channel allocation protocol}
\subtitle{Identifying connected groups of access points}
\author{Hans Jørgen Furre Nygårdshaug}

\begin{document}
\duoforside[dept={Institutt for informatikk},
  program={Informatikk: programmering og nettverk},
  long]

\chapter{Disposition} 
\section {Introduction part}
\begin{itemize} 
	\item Introduction to the wifi interference problem
	\item Taking a step back and looking at other attempts on solving the problem
	\item Begin presenting Torleiv and Magnus work, the idea, and maybe the p2p protocol.
	\item End of by showing that there is a problem with creating, limiting and updating groups.
\end{itemize}

\section {Main thesis part}
\subsection{The problem of data replicaton} 
\begin{itemize} 
	\item The problem.
	\item Possible solutions? References.
	\item Complexity, out of scope for thesis. Assume problem is solved.
\end{itemize} 
\subsection{The algorithm itself} 
\begin{itemize} 
	\item Elaborating on the problem, introducing the first algorithm suggestion.
	\item Explain simulation data creation with stochastic uniform distribution.
	\item Show how the group creation algorith was created, design decisions (iterations etc).
	\item Results with visualizations through the visualization tool. 
	\item Evaluate results and consider improvements. How will this work in the wild?
	\item Introduce SSB data, the data format and why it is relevant. How is the tool made.
	\item Same procedure with result visualization and result evaluation. Do we still need improvements?
	\item Introducing Wigle as data source. Show results on map?
\end{itemize}

\section {Concluding part}
\begin{itemize} 
	\item Have we created a meaningful algorithm that can be implemented in hardware?
\end{itemize}

\clearpage
\chapter{Introduction}
\section{Channel allocation} 
To deal with the problem of channel allocation we will think of an AP as a vertex in a graph. When an AP scans its radio
it can hear the strength of all nearby wireless networks measured in dBm (decibel milliwatts). This decibel value will be
the value of the edge between one AP to another. With a graph expressing the wireless network topology, the problem
of optimally distributing channels between APs boils down to a graph coloring problem. The number of colors in the color problem,
represents the number of non-overlapping channels in 802.11. Exactly how an algorithm can be designed to optimally distribute channels within the
interfering topology is out of the scope of this thesis. However we can define some invariants that has to be true
for such an algorithm to work:
\begin{enumerate} 
	\item All APs has to run the same algorithm
	\item All APs must run the algorithm on the same connected group
	\item Because of the complexity of the problem the algorthm must solve, the number of APs in the connected group can not be too big
\end{enumerate}

Point 1 is trivial to solve or mitigate, as only APs running the algorithm will actively participate in the channel selection. A simple way to make sure that the
same algorithm is used, is by having a software version that is consistently checked with the other APs in the connected group.

Point 2 and point 3 is will be the main focus of the rest of the master thesis, as these are not so easily solved.

We can define a wireless topology graph as a set of wireless APs that are grouped together and share information about their neighbours and interference levels.
This set is what will now on be referred to as a \textit{connected group.} All members of the connected group will be considered when running the channel assignment algorithm.
For the connected group to have an actual impact on the quality of a network connection, it has to consist of nodes that normally disturbs each other substantially.

An ideal example of a connected group is an apartment builiding. The channel allocation protocol lets APs share information about who-disturbs-who the most in the building.
Then each AP can run the channel allocation algorithm. Because they run it on the same graph, every AP will find the same optimal channel distribution throughout the building,
and then switch to the correct channel. 

Even though an apartment building is most likely an optimal delimination of a connected group, in reality creating such a group is a bigger challenge. As the whole channel allocation
protocol is based on decentralized peer-to-peer technology, and no centralized server with access to demographical and geographical divisions exists, the protocol will
have to discover suitable connected groups on its own. Moreover, when the group is created the protocol will have to replicate data so that
all participants of the group has all the data required to perform channel allocation. It will also need a way to make sure that the image of the current group
is consistent within all APs in the connected group. 
\chapter{Connected groups}
To enable collaborative channel allocation, it is paramount for every AP to know which other APs it is collaborating with.
One way to share information about who collaborates with who, is to let the access points group together. Information relevant for channel
allocation can then be shared freely within the group, between APs.

We will proceed by looking at some of the requirements for a group creation algorithm. It should work decentralized in a distributed fashion.
Hence, not only does the APs have to be imposed group membership, but they also have to be able to create and definine meaningful groups on their own.
Later we will propose an algorithm to create groups, and then evaluate computed groups based on the algorithm.


\section{Definition and criteria}
A connected group is a set of access points that are within close geographical range. The group should consist of APs that
interfere with each other when their channels are overlapping, so that overlap can avoided with a channel allocation algorithm run within the group.
Not all APs in the connected group will necessarily be able to hear each other on the radio directly,
but all nodes should be able to hear each other through a transitive relation (neighbour of a neighbour, etc.).

Some of the criteria for how groups should be created are imposed by the nature of P2P technology. For instance, the creation of groups could potentially
be easier if a centralized entity had the full picture of APs and their radio readings. It could then combine geographical information along with
AP readings to create consistent groups. However, we are going to be looking at how APs can organize themselves without a central coordinator.


\section{The Group Algorithm} 


\subsection{Simulation data}
Primarily, before beginning to implement and test the group creation algorithm, the task
is to create usable data to perform testing on. . Additionally, each node should have a neighbour list,
containing the respective interference measured in $-dBi$ for each neighbour. In addition,
the following parameters should be variable depending on each test scenario:

\begin{itemize}
\item Map size (width of x- and y-axis).
\item Number of nodes
\item Minimum distance between nodes (in meters)
\item Minimum measured $-dBi$ for a neighbour to consider it interfering
\end{itemize}

The program that generates the data is written in python, and can export the topology data to
JSON-format so it can be visualized or used by other applications.

The interference levels between APs is calculated by iterating through each node.
For each node $N$ we record its x and y position, and then start a second iteration through the nodes.
For each node in the second iteration $n$ we calculate the distance $d$ in
meters between $N$ and $n$ using Euclidean distance.



\printbibliography
\end{document}
