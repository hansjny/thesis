\chapter*{Preface}

\chapter*{Abstract}
Co-channel interference is the most significant degrader of quality of service in 802.11 Wi-Fi in dense residential aras.
The amount of observed neighbouring networks can grow as large as 20-30 networks in a modern apartment building. When
the interference from neighbouring networks degrades the perceived quality of service to the point where private, consumer networks no longer can provide
anywhere near the throughput guaranteed by the ISPs' service level agreements, this poses a problem that needs attention. 
In this thesis we argue that the problem would be a lot easier to solve if access points in residential areas had the opportunity to cooperate and coordinate
the distribution of channels within an enclosed group of high-impact neighbours. We proceed by gathering data of network topologies that can be used to facilitate simulations of such group creation, using different methods.
The groups has to contain access points that impacts each other highly to be of any value, so we treat the problem of identifying these groups as a distributed clustering problem,
where each node can only know about its immediate neighbours. A method of defining these clusters is suggested, simulated using network topologies, evaluated and modified.
Issues like changing network topologies are also addressed, through the concept of group splitting. Lastly, we suggest how an abstract protocol architecture that can faciliate group creation
in the real world might look, and what other research it could rely upon. 
